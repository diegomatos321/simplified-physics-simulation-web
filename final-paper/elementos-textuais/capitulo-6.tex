\chapter{Resposta a colisão}

A resposta a colisões é uma etapa fundamental em qualquer sistema de simulação física interativa. Após detectar que dois corpos estão em interpenetração, torna-se necessário aplicar um conjunto de correções que restaurem a plausibilidade física do movimento, evitando instabilidades numéricas. Neste capítulo discutimos os princípios clássicos, as formulações modernas e a relação direta entre métodos geométricos empregados na fase de detecção — como o vetor de translação mínima (MTV) — e métodos baseados em partículas e restrições, como o modelo proposto por \citeonline{jakobsen2001advanced}.

\section{Fundamentos da Resposta a Colisão}

A resposta a colisões consiste, essencialmente, em duas operações principais:

\begin{enumerate}
	\item \textbf{Correção de Posição}: eliminar a interpenetração entre dois corpos.
	\item \textbf{Correção de Velocidade}: remover ou ajustar componentes da velocidade que induziriam novo contato imediato.
\end{enumerate}

Embora os motores tradicionais realizem ambas as etapas, muitos sistemas baseados em PBD (\textit{Position-Based Dynamics}) focam totalmente na correção de posições, derivando velocidades implicitamente a partir da diferença entre posições sucessivas.

\section{Métodos Dinâmicos na Simulação Física}

A literatura apresenta diversas abordagens para modelar o movimento de corpos rígidos e deformáveis. Embora este trabalho se baseie no método simplificado de Jakobsen, é importante contextualizar outras categorias amplamente utilizadas em motores físicos modernos: métodos \textit{impulse-based}, métodos \textit{penalty-based} e métodos \textit{constraint-based} via multiplicadores de Lagrange.

\subsection{Método do Impulso}

Os métodos \textit{Método do Impulso} tratam colisões aplicando impulsos instantâneos que alteram diretamente as velocidades dos corpos para preservar o momento linear e angular. Essa abordagem é amplamente descrita em \citeonline{baraff1997rigid} e utilizada em motores como Havok e Bullet. O impulso é calculado em função da velocidade relativa no ponto de contato, resultando em um método eficiente para simulações em tempo real.

\subsection{Método de Penalidades}

Nos métodos \textit{Metodo de Penalidades}, colisões são tratadas como interpenetrações que geram forças de repulsão proporcionais à profundidade de penetração. Essas forças geralmente seguem modelos de mola e amortecimento, como apresentado por \citeonline{wyvill1986soft}. Trata-se de um método simples, porém sensível à escolha dos parâmetros de rigidez, podendo causar instabilidade numérica.

\subsection{Método Constraint-Based com Multiplicadores de Lagrange}

Os métodos baseados em restrições formulam os contatos como equações que devem ser satisfeitas exatamente. São resolvidos usando multiplicadores de Lagrange, como descrito em \citeonline{baraff1998large} e \citeonline{muller2007position}. Essa abordagem é robusta e adequada para sistemas complexos, mas exige a solução de sistemas lineares, tornando sua aplicação onerosa em plataformas Web.

\section{Processo de Separação}

A metodologia proposta por \citeonline{jakobsen2001advanced}, embora descrita em termos de integração Verlet e relaxamento de restrições geométricas, pode ser fundamentalmente entendida como uma precursora direta das modernas abordagens de \textit{Position-Based Dynamics} (PBD), popularizadas por \citeonline{muller2007109}. Esta conexão é crucial para contextualizar a relevância do método simplificado no panorama atual das simulações interativas.

A principal característica que une as duas abordagens é a ênfase na manipulação direta das posições das partículas para satisfazer restrições, contornando a complexidade da formulação tradicional baseada em forças ou impulsos.

A resposta à colisão envolve dois passos principais. O primeiro consiste em separar os elementos geométricos (vértices, arestas ou faces) que se encontram em interpenetração, o que caracteriza um processo estritamente geométrico. O segundo passo corresponde a um processo iterativo de relaxamento, no qual os elementos afetados ajustam suas posições de acordo com as restrições impostas pelo sistema físico.

Para dois objetos convexos $A$ e $B$ em colisão, o esquema de detecção de colisão deve retornar os pontos de contato de cada objeto e o tamanho da penetração. Com essas informações devemos tratar duas configurações possíveis: ponto de contato pertence a um vértice ou pertence a uma aresta.

\textbf{INSERIR DIAGRAMA MOSTRANDO OS DOIS CASOS}

No primeiro caso basta mover o vértice fora da região inválida. Já para o segundo caso o ponto de contato $p$ cai entre dois vértices $x_1$ e $x_2$ e nosso objetivo é corrigir as suas posições para configuração válida $x_1^*$ $x_2^*$, logo pela equação da reta ele pode ser descrito como uma combinação linear

\begin{equation}
	p = \alpha x_1 + (alpha-1) x_2
\end{equation}

Dessa forma Jakobsen computa as novas posições como

\begin{align}
	
\end{align}

%Para dois objetos convexos $A$ e $B$ em colisão, o processo de separação requer três parâmetros principais: os pontos de contato $p_A$ de $A$, os pontos de contato $p_B$ de $B$ e o ponto de projeção $q$, assumido pela simulação como o ponto em que ambos os objetos estão em contato. O ponto de projeção coincide com o plano de separação $H(\vec{v}, \delta)$ definido na geometria local, de forma que, como regra geral, basta mover os pontos de $A$ ou $B$ na direção desse plano para remover a interpenetração.

\section{Algoritmo de Expansão de Politopos (EPA)}

Para realizar a separação de dois objetos usando o algoritmo SAT basta calcularmos o MTV como visto na seção \ref{sec:sat}. Já para o GJK é preciso fazer um segundo passo, uma extensão do algoritmo que nos permite encontrar a normal correta e profundidade das colisões.

O Algoritmo de Expansão de Politopos (do inglês Expanding Polytope Algorithm, EPA) cria um polítopo (ou polígono) dentro da Diferença de Minkowski e iterativamente expandi-lo até atingirmos a borda da Diferença de Minkowski. EPA executa essa tarefa utilizando a mesma função de suporte utilizada nos demais algoritmos e a mesma noção de um simplex.

Este algoritmo é uma extensão porque sua entrada é o Simplex final do GJK que contém a origem e encontra o MTV. A distância entre o ponto mais próximo com a origem é a profundidade de penetração ($\delta$). Além disso, o vetor normal para o ponto mais próximo é a direção de separação (ponto de contato). A solução ingênua é usar o normal da face mais próxima da origem, porém um simplex não precisa conter nenhuma das faces do polígono original, o quê pode acabar com uma normal incorreta.

O algoritmo expande o Simplex adicionando vértices a ele até encontrarmos a normal mais próxima de uma face que está no polígono original.

\begin{algorithm}[H]
	\caption{EPA}
	\KwIn{Simplex}
	\KwOut{separation $v$, penetration $\delta$}
	
	\For{$i \leftarrow 0$ \KwTo $i < MAX\_ITERATION$}{
		e $\leftarrow$ Encontrar aresta mais próxima a origem \\
		p $\leftarrow$ Calcular novo ponto de suporte na direção da normal de e \\
		$\delta \leftarrow p \cdot normal(e)$ \\
		\If{$|\delta - length(e)| < TOLERANCE$}{
			\KwRet{normal(e), $\delta$}
		}
		
		Adicionar ponto ao simplex
	}
\end{algorithm}

É importante limitar o número de iterações para evitar que a rotina entre em loop infinito em casos degenerados, como esse algoritmo converge rapidamente uma constante igual a 30 é um bom limite superior. Matematicamente a distância deve ser igual a zero, mas por conta da artimética de ponto flutuante, uma tolerância pequena deve ser aceita, como $10^{-3}$.

\section{Desafios e Limitações das Abordagens de Animação Física Simplificada}
\label{sec:limitacoes}

Embora métodos de animação física simplificada sejam eficazes para aplicações interativas na Web, eles apresentam limitações e desafios inerentes às aproximações empregadas. Como discutido anteriormente, o objetivo principal dessas técnicas não é fidelidade física absoluta, mas sim a geração de comportamentos verossímeis, estáveis e visualmente plausíveis. No entanto, essa busca por simplicidade e desempenho resulta em compromissos importantes.

\subsection{Jitter e Distribuição de Correções}

Em sistemas baseados em posições — incluindo métodos inspirados em Jakobsen e frameworks modernos como PBD — a estabilidade depende fortemente do processo de correção de posições. Um dos problemas mais recorrentes é o \textbf{jitter}, um tremor ou oscilação indesejada no posicionamento dos corpos, especialmente perceptível quando múltiplas restrições são aplicadas simultaneamente ou quando o sistema é altamente rígido.

Outro aspecto crítico é o \textbf{tuning da distribuição de correções}: decidir como cada restrição contribui para o deslocamento final dos pontos ou corpos envolvidos. Distribuições mal balanceadas podem causar:

\begin{itemize}
	\item instabilidade numérica,
	\item objetos que ``vazam'' através de outros,
	\item corpos que recebem correções excessivas e oscilam,
	\item propagação exagerada de energia através da cadeia de restrições.
\end{itemize}

A calibração ideal dessa distribuição varia com o tipo de geometria, número de restrições, massa relativa dos objetos e o passo temporal da simulação. Isso torna o ajuste fino (\textit{tuning}) uma tarefa empírica e frequentemente dependente de tentativa e erro.

\subsection{Falhas Comuns: Empilhamento, Tunneling e Jittering}

Sistemas simplificados frequentemente apresentam falhas clássicas observadas em simulações físicas:

\begin{itemize}
	\item \textbf{Empilhamento instável}: métodos simplificados têm dificuldade em manter pilhas estáveis de objetos, principalmente quando as correções não são distribuídas de forma global e consistente. O empilhamento tende a ``escorregar'' ou colapsar devido à falta de amortecimento numérico adequado.
	
	\item \textbf{Tunneling}: ocorre quando objetos em alta velocidade atravessam outros sem detectar colisão. Métodos baseados exclusivamente em detecção discreta apresentam maior risco, especialmente quando o passo temporal é grande ou a geometria é fina.
	
	\item \textbf{Jittering}: pequenos movimentos involuntários causados por correções excessivas, flutuações numéricas ou conflito entre múltiplas restrições. Este problema é particularmente visível em objetos apoiados no chão, que parecem tremer continuamente.
\end{itemize}

Apesar de existirem soluções como CCD (\textit{Continuous Collision Detection}) ou múltiplas iterações de estabilidade, tais técnicas aumentam o custo computacional e podem não se alinhar ao objetivo da animação simplificada.

\subsection{Limitações da Abordagem Simplificada}

Ao adotar métodos simplificados, abre-se mão de características essenciais de motores físicos completos. Entre as limitações mais relevantes estão:

\begin{itemize}
	\item \textbf{Ausência de conservação precisa de energia e momento}, o que reduz o realismo de certas interações.
	\item \textbf{Incapacidade de simular materiais complexos} (ex.: fricção anisotrópica, torques realistas, elasticidade avançada).
	\item \textbf{Dependência de parâmetros empíricos}, sem interpretação física clara.
	\item \textbf{Menor robustez para geometrias arbitrárias}, especialmente polígonos concavos ou mal escalonados.
	\item \textbf{Dificuldade de lidar com sistemas altamente conectados} (estruturas rígidas, máquinas, esqueletos).
\end{itemize}

Essas limitações não invalidam o uso das técnicas, mas reforçam a necessidade do uso da aplicação a cenários onde a prioridade é a responsividade, e não a precisão física.
