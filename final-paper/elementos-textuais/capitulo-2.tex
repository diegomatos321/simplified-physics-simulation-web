\chapter{Animação Baseada em Física}

A animação baseada em física é uma abordagem de geração de movimento que aparenta seguir princípios físicos básicos, mesmo que as equações envolvidas sejam tratadas de forma aproximada ou altamente simplificada. Diferentemente da animação tradicional, na qual o animador define manualmente posições e rotações ao longo do tempo (\textit{keyframe animation}), a animação física permite que o comportamento dos objetos emerja naturalmente das forças, restrições e interações entre os corpos simulados.

Nesse contexto, a animação física simplificada busca um equilíbrio entre precisão e desempenho: modelos matemáticos são utilizados como guia, mas a prioridade está na estabilidade visual e na resposta interativa. Diferentemente da simulação científica, onde precisão e correção numérica são cruciais, na animação para fins gráficos ou interativos \textbf{o objetivo não é obter resultados fisicamente corretos, mas sim um comportamento verossímil}. O foco está em transmitir sensação de peso, inércia e colisões de maneira convincente ao usuário, mesmo quando obtidos por heurísticas.

\section{Conceitos e Definições}

O objetivo central da animação baseada em física é resolver numericamente as equações que descrevem o movimento de objetos em um mundo virtual. Tais equações derivam das leis fundamentais da mecânica clássica, formuladas por Isaac Newton.

\begin{quote}
	\textbf{Primeira lei de Newton.}
	Na ausência de forças externas, um objeto em repouso permanece em repouso e um objeto em movimento continua em movimento com velocidade constante. Apenas forças externas podem alterar o estado de movimento.
\end{quote}

\begin{quote}
	\textbf{Segunda lei de Newton.}
	Para um corpo de massa constante $m$ submetido a uma força $\vec{F}$, o movimento é descrito por:
	\begin{equation}
		\vec{F} = m\vec{a} = m\frac{d\vec{v}}{dt} = m\frac{d^2\vec{x}}{dt^2}.
	\end{equation}
\end{quote}

\begin{quote}
	\textbf{Terceira lei de Newton.}
	Para toda força exercida em um corpo existe uma força de igual magnitude e direção oposta exercida no corpo que a gerou.
\end{quote}

Integrando a aceleração ao longo do tempo obtêm-se velocidade e posição, que determinam a trajetória dos objetos. Além das forças, a simulação deve também considerar colisões, atrito, restituição e restrições entre corpos (como juntas). Em implementações simplificadas, esses efeitos são aproximados por regras empíricas e técnicas numéricas que priorizam eficiência computacional.

De maneira geral, um ciclo de simulação física engloba:
\begin{enumerate}
	\item coleta e soma das forças aplicadas (gravidade, vento, atrito etc.);
	\item integração temporal das equações de movimento;
	\item detecção e resposta a colisões;
	\item atualização das posições e posterior renderização.
\end{enumerate}

O método descrito por \citeonline{jakobsen2001advanced} reúne um conjunto de técnicas simples, estáveis e eficientes, que permitem obter resultados visualmente satisfatórios mesmo com aproximações físicas significativas. Seu algoritmo iterativo permite aumentar a precisão ao custo de desempenho, ajustando dinamicamente essa relação conforme a necessidade. O sucesso da abordagem se deve à combinação entre o integrador Verlet, a solução de restrições por relaxamento e uma estratégia de resolução de colisões baseada em projeção. A seguir, descrevem-se os principais componentes desse método.

\section{Representação de Corpos Rígidos}

Um \textbf{corpo rígido} é um objeto cuja forma e volume permanecem invariáveis durante a simulação. Em termos matemáticos, a distância entre quaisquer dois pontos do corpo é constante, independentemente das forças aplicadas. Essa suposição simplifica o problema, permitindo representar o corpo apenas por grandezas globais: posição, orientação e velocidades linear e angular.

A representação matemática de um corpo rígido é dada por:
\begin{itemize}
	\item \textbf{Posição} $\vec{p}$: coordenadas do centro de massa;
	\item \textbf{Orientação} $R$: matriz de rotação ou quaternion;
	\item \textbf{Velocidade linear} $\vec{v}$: variação temporal da posição;
	\item \textbf{Velocidade angular} $\vec{\omega}$: variação temporal da orientação;
	\item \textbf{Massa} $m$ e \textbf{tensor de inércia} $I$: medidas de resistência à aceleração.
\end{itemize}

A dinâmica translacional e rotacional é governada pelas equações:
\begin{align}
	m \cdot \frac{d\vec{v}}{dt} &= \sum \vec{F}, \\
	I \cdot \frac{d\vec{\omega}}{dt} &= \sum \vec{\tau},
\end{align}

onde $\sum \vec{F}$ é o somatório das forças externas e $\sum \vec{\tau}$ o somatório dos torques. Em simulações mais simples — como tecidos, cordas ou partículas — a rotação é frequentemente ignorada, reduzindo a complexidade computacional.

\section{Dinâmica de Partículas}

A \textbf{dinâmica de partículas} é uma abordagem na qual o sistema é composto por partículas independentes. Cada partícula possui posição, velocidade e massa, e suas interações são modeladas por forças (como gravidade ou molas) ou por restrições geométricas (como manter distâncias constantes).

O estado de uma particula no instante $t$ é dado por:
\[
X(t) =
\begin{pmatrix}
	x(t) \\
	v(t)
\end{pmatrix}.
\]

Seja $F(t)$ a soma das forças que atuam sobre a partícula e $m$ sua massa. O movimento pode ser descrito por:
\begin{equation}
	\frac{d}{dt} X(t)
	=
	\frac{d}{dt}
	\begin{pmatrix}
		x(t) \\
		v(t)
	\end{pmatrix}
	=
	\begin{pmatrix}
		v(t) \\
		\dfrac{F(t)}{m}
	\end{pmatrix}.
\end{equation}

A dinâmica de partículas é amplamente utilizada para simular tecidos, fluidos e efeitos visuais (como fumaça, poeira ou explosões) devido à sua flexibilidade e à capacidade de gerar comportamentos complexos emergentes a partir de regras simples.

