\chapter{Animação baseado em física}

A animação baseada em física é uma abordagem para gerar movimento em ambientes virtuais a partir da aplicação de leis físicas simplificadas. Diferentemente da animação tradicional, na qual o animador define manualmente as posições e rotações dos objetos ao longo do tempo (\textit{keyframe animation}), a animação física permite que o comportamento do sistema surja de forma natural como resultado das forças, restrições e interações entre os corpos simulados.

Essa técnica é amplamente utilizada em jogos eletrônicos, simulações científicas, filmes e aplicações interativas, pois permite reproduzir movimentos realistas e dinâmicos, adaptando-se automaticamente a mudanças no ambiente. No entanto, ela exige uma modelagem matemática cuidadosa e algoritmos eficientes para garantir estabilidade e desempenho em tempo real.

Na maioria dos casos, é preciso usar métodos sofisticados e exatos para simular a dinâmica. Porém, em aplicações de jogos eletrônicos, a precisão não é realmente o mais importante, mas sim que o resultado final tenha uma aparência realista e que possa ser executado rapidamente. 

\section{Conceitos e Definições}

O objetivo principal da animação baseada em física é resolver numericamente as equações que descrevem o movimento dos objetos em um mundo virtual. Essas equações derivam das leis fundamentais da mecânica clássica, formuladas por Isaac Newton

\begin{quote}
	\textbf{Primeira lei de Newton.} 
	Na ausência de forças externas, um objeto em repouso permanece em repouso, e se o objeto está em movimento ele permanece em movimento com velocidade constante. 
	Isto é, só forças externas podem mudar o movimento de um objeto.
\end{quote}

\medskip

\begin{quote}
	\textbf{Segunda lei de Newton.} 
	Para um corpo de massa constante $m$ experimentando uma força $\vec{F}$, o movimento do corpo sobre o tempo é dado por:
	
	\begin{equation}
		\vec{F} = m \vec{a} = m \frac{d\vec{v}}{dt} = m \frac{d^2\vec{x}}{dt^2}
	\end{equation}
\end{quote}

\medskip

\begin{quote}
	\textbf{Terceira lei de Newton.} 
	Se aplicada uma força externa sobre um objeto, há uma força de igual magnitude mas em direção oposta exercida sobre o causador da força. 
	Isto é, a toda ação corresponde uma reação.
\end{quote}

onde $\vec{F}$ representa a força resultante aplicada a um corpo, $m$ é a sua massa e $\vec{a}$ é a aceleração resultante.  
Integrando essa aceleração ao longo do tempo, obtêm-se a velocidade e a posição dos objetos, que determinam sua trajetória e comportamento.

A simulação física deve considerar também aspectos como colisões, atrito, restituição e restrições entre corpos (por exemplo, juntas ou conexões). Em implementações simplificadas, esses efeitos são representados por aproximações numéricas e regras empíricas que buscam equilibrar precisão e desempenho.

Do ponto de vista computacional, um sistema de animação física pode ser descrito pelo ciclo:
\begin{enumerate}
	\item Coleta de forças aplicadas (como gravidade, vento, empuxo);
	\item Integração temporal das equações de movimento;
	\item Detecção e resposta a colisões;
	\item Atualização das posições e renderização dos objetos.
\end{enumerate}

\citeonline{jakobsen2001advanced} apresentou um conjunto de métodos e técnicas que, unidas, conseguem atingir em grande parte estes objetivos. Estes métodos são relativamente simples de implementar (comparados com outros esquemas) e são rápidas e estáveis. O algoritmo é iterativo e permite aumentar a precisão do método sacrificando parte de sua rapidez: se uma pequena fonte de imprecisão é aceita, o código pode conseguir uma execução mais rápida. Esta margem de erro ainda pode ser ajustada de forma adaptativa em tempo de execução.

Em geral, o sucesso deste método vem da combinação de varias técnicas que se beneficiam umas das outras, principalmente o uso do integrador Verlet, o tratamento de colisões usando projeção, e a resolução de restrições de distância usando relaxamento. A seguir serão descritas as componentes mais importantes da abordagem proposta por Jakobsen.

\section{Representação de Corpos Rígidos}

Um \textbf{corpo rígido} é um objeto cuja forma e volume não se deformam durante a simulação. Ou seja, a distância entre quaisquer dois pontos do corpo permanece constante, independentemente das forças aplicadas. Essa suposição simplifica o cálculo do movimento, permitindo que o corpo seja descrito apenas por sua posição, orientação, velocidade linear e velocidade angular.

A representação matemática de um corpo rígido envolve:
\begin{itemize}
	\item \textbf{Posição} $\vec{p}$: coordenadas do centro de massa no espaço;
	\item \textbf{Orientação} $R$: matriz de rotação ou quaternion que define a orientação do corpo;
	\item \textbf{Velocidade linear} $\vec{v}$: taxa de variação da posição;
	\item \textbf{Velocidade angular} $\vec{\omega}$: taxa de variação da orientação;
	\item \textbf{Massa} $m$ e \textbf{tensor de inércia} $I$: determinam a resistência à aceleração linear e angular, respectivamente.
\end{itemize}

A dinâmica de um corpo rígido é regida pelas equações diferenciais:
\begin{align}
	m \cdot \frac{d\vec{v}}{dt} &= \sum \vec{F} \\
	I \cdot \frac{d\vec{\omega}}{dt} &= \sum \vec{\tau}
\end{align}

onde $\sum \vec{F}$ representa a soma das forças externas e $\sum \vec{\tau}$ o somatório dos torques aplicados.  
Essas equações descrevem o movimento translacional e rotacional dos corpos, respectivamente.

Em simulações simplificadas (como tecidos, cordas ou partículas), é comum ignorar a rotação e modelar apenas o movimento linear, o que reduz significativamente a complexidade computacional.
\section{Dinâmica de Partículas}

A \textbf{dinâmica de partículas} é uma abordagem alternativa à simulação de corpos rígidos, na qual um sistema é representado como um conjunto de partículas independentes. Cada partícula possui propriedades como posição, velocidade e massa, e as interações entre elas são definidas por restrições ou forças (como gravidade, molas ou colisões).

O estado de uma partícula no instante t é descrito por sua posição x(t) e sua velocidade v(t), e é representado por um vetor X(t) da forma

$$X(t) = \begin{pmatrix}
	x(t) \\
	v(t)
\end{pmatrix} $$

Além disso para simular o movimento de uma partícula é necessário conhecer as forças que agem nela no instante t. Seja m a massa da partícula e $F(t)$ a soma de todas as forças que agem na a partícula: gravidade, vento, atrito, etc. O movimento da partícula pode ser definido como:

\begin{equation}
	\frac{d}{dt} X(t) = \frac{d}{dt} \begin{pmatrix}
		x(t) \\
		v(t) \\
	\end{pmatrix} = \begin{pmatrix}
		v(t) \\
		\frac{F(t)}{m}
	\end{pmatrix}
\end{equation}

O método de partículas é amplamente utilizado em simulações de tecidos, fluidos e efeitos visuais (fumaça, explosões, poeira), pois permite representar comportamentos complexos emergentes a partir de regras simples de interação.
