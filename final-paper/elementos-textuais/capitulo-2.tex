\chapter{Animação baseado em física}

Na maioria dos casos, é preciso usar métodos sofisticados e exatos para simular a dinâmica. Porém, em aplicações de jogos interativos, a precisão não é realmente o mais importante, mas sim que o resultado final tenha uma aparência realista e que possa ser executado rapidamente. 

Jakobsen [24] apresentou um conjunto de métodos e técnicas que, unidas, conseguem atingir em grande parte estes objetivos. Estes métodos são relativamente simples de im-
plementar (comparados com outros esquemas) e têm um bom desempenho. O algoritmo é iterativo e permite aumentar a precisão do método sacrificando parte de sua rapidez: se
uma pequena fonte de imprecisão é aceita, o código pode conseguir uma execução mais rápida. Esta margem de erro ainda pode ser ajustada de forma adaptativa em tempo de
execução.

Em geral, o sucesso deste método vem da combinação de varias técnicas que se beneficiam umas das outras, principalmente o uso do integrador Verlet, o tratamento de colisões usando projeção, e a resolução de restrições de distância usando relaxamento. A seguir serão descritas as componentes mais importantes da abordagem proposta por Jakobsen.

\section{Integrador Verlet}

\section{Restrição Linear}

\section{Restrição Angular}

\section{Restrição de Revolução}

\section{Resolvendo restrições concorrentes por relaxamento}
