\chapter{Otimizações}

Como qualquer objeto pode potencialmente colidir com qualquer outro objeto, uma simulação com $n$ objetos requer $(n-1)+(n-2) + \dots + 1 = n(n-1)/2 = O(n^2)$ testes de pares, no pior caso. Devido à complexidade de tempo quadrática, testar ingenuamente cada par de objetos para verificar colisões rapidamente se torna muito caro, mesmo para valores moderados de n. 

Reduzir o custo associado ao teste de pares afetará o tempo de execução apenas linearmente. Para realmente acelerar o processo, o número de pares testados deve ser reduzido. Essa redução é realizada separando o tratamento de colisões de múltiplos objetos em duas fases: Narrow Phase e Broad Phase.

\section{Broad Phase}

Segundo \citeonline{ericson2004real}, o princípio fundamental da otimização é que nada é mais rápido do que não ter que realizar uma tarefa. Dessa forma, as melhores otimizações para acelerar uma rotina giram em torno de reduzir o trabalho ao mínimo possível o mais cedo possível.

A Broad Phase identifica grupos menores de objetos que podem estar colidindo e rejeita rapidamente aqueles que não estão. Como os objetos só podem atingir coisas que estão próximas a eles, os testes contra objetos distantes podem ser evitados dividindo as coisas espacialmente.

A consulta de colisão mais simples é o problema teste de interseção: responder à pergunta booleana se dois objetos (estáticos), A e B, estão se sobrepondo em suas posições e orientações dadas. As consultas de interseção booleanas são rápidas e fáceis de implementar e, portanto, são ideias para Broad Phase.

\subsection{Divisão espacial em grade uniforme}

As técnicas de particionamento espacial são ótimas candidatas para Broad Phase, dividindo o espaço em regiões e verificando se os objetos se sobrepõem à mesma região do espaço. Como os objetos só podem se interceptar se se sobrepuserem à mesma região do espaço, o número de testes de pares de objetos é drasticamente reduzido.

Um esquema muito eficaz de subdivisão espacial consiste em sobrepor um espaço com uma grade regular. Essa grade divide o espaço em várias células de tamanho igual. Cada objeto é então associado às células com as quais se sobrepõem.

Devido à uniformidade da grade, acessar uma célula correspondente a uma determinada coordenada é simples e rápido: os valores das coordenadas do mundo são simplesmente divididos pelo tamanho da célula para obter as coordenadas da célula. Dadas as coordenadas de uma célula específica, localizar as células vizinhas também é trivial.

Em termos de desempenho, um dos aspectos mais importantes dos métodos baseados em grade é a escolha de um tamanho de célula apropriado. Existem quatro questões relacionadas ao tamanho da célula que podem prejudicar o desempenho:

\begin{enumerate}
	\item Se as células forem muito pequenas, um grande número de células precisará ser atualizado.
	\item Se os objetos forem pequenos e as células da grade forem grandes, haverá muitos objetos em cada célula.
	\item Se os objetos tiverem uma geometria muito complexa isso irá afetar os testes de interseção, eles devem ser divididos em partes menores.
	\item É possível que os objetos tenham ambas características anteriores. Sendo necessário outra abordagem como grade hierárquicas.
\end{enumerate}

\section{Narrow Phase}

A fase restrita consiste nos testes pareados dentro dos subgrupos. Ela é responsável por determinar as colisões exatas, se houver.

\section{Separando simulação da Thread principal}
