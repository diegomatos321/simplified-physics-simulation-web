\chapter{O Fecho Convexo}

O \textit{Fecho Convexo} (FC) de um conjunto é definido como a menor região convexa que contém todos os seus pontos, sendo frequentemente utilizado como um volume delimitador. Determinar o fecho convexo é um problema recorrente em computação geométrica, especialmente quando se deseja organizar pontos em estruturas mais simples ou acelerar operações posteriores, como testes de colisão.

Em sistemas de simulação física e detecção de colisões, a representação geométrica dos objetos influencia diretamente a eficiência dos cálculos. Formas complexas, com muitos vértices ou superfícies não convexas, tornam tais testes significativamente mais custosos. Uma solução comum é empregar aproximações convexas ou volumes delimitadores que possibilitam testes rápidos sem sacrificar excessivamente a precisão da simulação.

A principal motivação para o uso de volumes delimitadores é que formas mais simples (como caixas ou esferas) permitem testes de sobreposição muito mais baratos do que a geometria original que envolvem. Dessa forma, o FC atua como um primeiro filtro: apenas quando o teste de interseção entre volumes delimitadores retorna positivo é que se procede para verificações mais detalhadas na geometria original. Em muitos casos, o próprio volume delimitador já é suficiente para caracterizar uma colisão.

\textbf{INSERIR FIGURA DO FECHO CONVEXO DE OBJETOS EM INTERSECÇÃO E SEM INTERSECÇÃO}

Segundo \citeonline{moller2018}, nem todos os objetos geométricos servem como volumes delimitadores eficazes. As propriedades desejáveis para volumes delimitadores incluem:

\begin{itemize}
	\item Testes de interseção de baixo custo
	\item Ajuste preciso
	\item Cálculo econômico
	\item Fácil de girar e transformar
	\item Consome pouca memória
\end{itemize}

\textbf{INSERIR IMAGENS COM TIPOS DIFERENTES FORMAS DE FECHO CONVEXO}

\section{Caixa Delimitadora Alinhada ao Eixo Coordenado (AABB)}

A caixa delimitadora mínima para um conjunto de pontos em $N$ dimensões é aquela que possui o menor volume possível e ainda assim contém todos os pontos. Notavelmente, a AABB mínima para um conjunto é a mesma que a AABB mínima de seu fecho convexo, fato útil em heurísticas para computação eficiente.

A caixa delimitadora alinhada aos eixos (Axis-Aligned Bounding Box, AABB) é um dos volumes delimitadores mais utilizados. Trata-se de um paralelepípedo (ou retângulo, em 2D) cujas faces são paralelas aos eixos do sistema de coordenadas. Seu teste de interseção é extremamente simples:

\begin{algorithm}[H]
	\caption{Teste de Interseção AABB}
	\KwIn{A e B: volumes AABB}
	\KwOut{Verdadeiro se houver colisão}
	
	\If{$A.x_{max} < B.x_{min}$ \textbf{ou} $A.x_{min} > B.x_{max}$}{
		\Return{False}
	}
	\If{$A.y_{max} < B.y_{min}$ \textbf{ou} $A.y_{min} > B.y_{max}$}{
		\Return{False}
	}
	\If{$A.z_{max} < B.z_{min}$ \textbf{ou} $A.z_{min} > B.z_{max}$}{
		\Return{False}
	}
	\Return{True}
\end{algorithm}

AABBs são eficientes, porém perdem precisão quando o objeto sofre rotações, pois a caixa permanece alinhada aos eixos globais.

\section{Caixa Orientada (OBB)}

Uma Caixa Orientada (Oriented Bounding Box, OBB) é uma caixa retangular que pode estar arbitrariamente rotacionada em relação aos eixos do sistema de coordenadas. É definida por um ponto central $c$, por três vetores ortogonais $\hat{u}_i$ que compõem sua orientação e por semi-extensões $e_i$:

\begin{equation}
	OBB = \left\{ c + \sum_{i=1}^{3} \alpha_i \hat{u}_i \; \middle| \; -e_i \le \alpha_i \le e_i \right\}
\end{equation}

OBBs geralmente oferecem melhor ajuste, especialmente para objetos alongados ou rotacionados, reduzindo falsos positivos. Porém, o teste de interseção é mais caro que o das AABBs, o que torna seu uso preferível em cenas com número reduzido de objetos ou em simulações nas quais a precisão de ajuste é particularmente importante.

\section{Esferas e Elipsoides}

As \textbf{esferas} constituem o volume delimitador mais simples, definidas apenas por um centro $c$ e um raio $r$:

\begin{equation}
	\text{Sphere} = \{ x \in \mathbb{R}^3 \; | \; \|x - c\| \le r \}
\end{equation}

Testes de colisão entre esferas são extremamente rápidos, porém inadequados para objetos de proporções irregulares. Elipsoides oferecem melhor ajuste, mas aumentam o custo de teste. Por isso, essas formas são frequentemente utilizadas em fases preliminares da detecção, ou como nós intermediários em hierarquias de volumes delimitadores (BVH).

\section{Quickhull}

O \textit{Quickhull} é um algoritmo para o cálculo do fecho convexo de um conjunto finito de pontos em qualquer dimensão, adotando uma estratégia de divisão e conquista semelhante ao \textit{quicksort} \cite{barber1996quickhull}.

O Quickhull parte de um conjunto de pontos $S$ e constrói o polígono (ou poliedro) convexo que os contém.  O processo para 2 dimensões pode ser descrito em linhas gerais da seguinte forma:

\begin{algorithm}[H]
	\caption{Quickhull 2D}
	\LinesNumbered
	\SetAlgoLined
	\KwIn{Polígono Convexo}
	\KwOut{Lista dos vértices do fecho convexo}
	
	Encontre os pontos de menor e maior coordenada em $x$; eles pertencem ao fecho convexo.\\
	Use a linha formada pelos dois pontos para dividir o conjunto em dois subconjuntos de pontos, que serão processados de forma recursiva. \\
	Para cada subconjunto, encontre o ponto mais distante da linha; ele forma um triângulo que exclui pontos interiores.\\
	Repita recursivamente os dois passos anteriores nas duas linhas formadas pelos dois novos lados do triângulo. \\
	O processo termina quando todos os subconjuntos estão vazios.
	\label{al:quickhull_2d}
\end{algorithm}

O Quickhull apresenta complexidade média $O(n \log n)$ em 2D, podendo chegar a $O(n^2)$ em casos degenerados. Em 3D, adapta-se a construções poliedrais mais complexas, mantendo o mesmo princípio recursivo.

