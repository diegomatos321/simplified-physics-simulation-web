\chapter{O Fecho Convexo}

O Fecho Convexo (FC) de um conjunto é definido como a menor região convexa que contenha todos os pontos (também chamado como volume delimitador). Achar o fecho convexo de um conjunto de pontos, por exemplo, é um problema que aparece quando queremos organizar o conjunto, agrupar os pontos em uma região simples. 

Em sistemas de simulação física e detecção de colisões, a representação geométrica dos objetos tem impacto direto na eficiência dos cálculos. Formas complexas, compostas por muitos vértices e superfícies não convexas, tornam os testes geométricos computacionalmente caros.  

Para contornar esse problema, é comum empregar aproximações convexas ou volumes delimitadores — denominados \textit{fechos convexos} — que permitem realizar testes mais rápidos sem comprometer significativamente a precisão da simulação.

\textbf{INSERIR FIGURA DO FECHO CONVEXO DE OBJETOS EM INTERSECÇÃO E SEM INTERSECÇÃO}

A ideia é que formas mais simples (como caixas e esferas) tenham testes de sobreposição mais baratos do que os objetos complexos que eles delimitam. O uso do FC permite testes rápidos de rejeição de sobreposição, pois só é necessário testar a geometria complexa delimitada quando a consulta inicial de sobreposição para os volumes delimitadores resulta em um resultado positivo. Em algumas aplicações, o próprio teste de interseção dos volumes delimitadores serve como prova suficiente de colisão.

Segundo \citeonline{moller2018}, nem todos os objetos geométricos servem como volumes delimitadores eficazes. As propriedades desejáveis para volumes delimitadores incluem:

\begin{itemize}
	\item Testes de interseção de baixo custo
	\item Ajuste preciso
	\item Cálculo econômico
	\item Fácil de girar e transformar
	\item Consome pouca memória
\end{itemize}

\textbf{INSERIR IMAGENS COM TIPOS DIFERENTES FORMAS DE FECHO CONVEXO}

A ideia principal por trás dos volumes delimitadores é preceder testes geométricos complexos com testes menos dispendiosos que permitem que o teste seja interrompido precocemente. Para suportar testes de sobreposição de baixo custo, o volume delimitador deve ter uma forma geométrica simples. Ao mesmo tempo, para tornar o teste de corte antecipado o mais eficaz possível, o volume delimitador também deve ser o mais ajustado possível, resultando em um trade-off entre o ajuste e o custo do teste de interseção.

\section{Caixa Delimitadora Alinhada ao Eixo Coordenado (AABB)}

A caixa delimitadora mínima para um conjunto de pontos em N dimensões é a caixa com a menor medida (área, volume, ou hipervolume em dimensões superiores) possível que englobe todos os pontos. A caixa delimitadora mínima de um conjunto de pontos é a mesma que a caixa delimitadora mínima de seu FC, um fato que pode ser usado de forma heurística para acelerar a computação.

A caixa delimitadora alinhada aos eixos (em inglês "axis-aligned bounding box", ou AABB) é um dos volumes delimitadores mais comuns. É um paralelepípedo (ou retângulo em 2D) cujas faces são paralelas aos eixos de coordenadas.

\begin{algorithm}[H]
	\caption{Teste de Intersecção AABB}
	\KwIn{Polígonos convexos A e B}
	\KwOut{Verdadeiro se estão colidindo}
	
	\If{$x_{max}$ de A < $x_{min}$ de B ou $x_{min}$ de A > $x_{max}$ de B}{
		\Return{False}
	}
	\If{$x_{max}$ de A < $x_{min}$ de B ou $x_{min}$ de A > $x_{max}$ de B}{
		\Return{False}
	}
	\If{$x_{max}$ de A < $x_{min}$ de B ou $x_{min}$ de A > $x_{max}$ de B}{
		\Return{False}
	}
	\Return{True}
\end{algorithm}

\section{Caixa Orientada (OBB)}

Uma \textbf{Oriented Bounding Box} (OBB) é uma caixa retangular que pode estar rotacionada em relação aos eixos globais.  
Ela é definida por um ponto central $c$, um conjunto de vetores ortogonais $\hat{u}_i$ representando a orientação, e comprimentos semi-extensões $e_i$ em cada direção:

\begin{equation}
	OBB = \left\{ c + \sum_{i=1}^{3} \alpha_i \hat{u}_i \; \middle| \; -e_i \le \alpha_i \le e_i \right\}
\end{equation}

OBBs geralmente fornecem um ajuste mais justo em torno de objetos alongados ou rotacionados, reduzindo falsos positivos de colisão.  
O custo computacional é maior que o de AABBs, mas compensado em simulações com poucos objetos ou com formas muito irregulares.

\section{Esferas e Elipsoides}

As \textbf{esferas} são os volumes mais simples de todos, definidas apenas por um centro $c$ e um raio $r$:

\begin{equation}
	\text{Sphere} = \{ x \in \mathbb{R}^3 \; | \; \|x - c\| \le r \}
\end{equation}

Elas permitem testes de colisão extremamente rápidos, mas representam mal objetos com proporções muito distintas.  
Por esse motivo, esferas e elipsoides são frequentemente empregadas apenas em fases preliminares de detecção, ou como volumes intermediários em hierarquias de limitação (Bounding Volume Hierarchies, BVH).

\section{Quickhull}

Quickhull é um algoritmo incremental para Fecho Convexo de um conjunto finito de pontos de qualquer dimensão. Ele usa uma abordagem de divisão e conquista semelhante à do quicksort, da qual seu nome deriva \cite{barber1996quickhull}.

O Quickhull parte de um conjunto de pontos $S$ e constrói o polígono (ou poliedro) convexo que os contém.  O processo para 2 dimensões pode ser descrito em linhas gerais da seguinte forma:

\begin{algorithm}[H]
	\caption{Quickhull 2D}
	\LinesNumbered
	\SetAlgoLined
	\KwIn{Polígono Convexo}
	\KwOut{Lista dos vértices que representam o fecho convexo}
	Encontre os pontos com coordenadas mínimas e máximas x, pois estes sempre farão parte do casco convexo. \\
	Use a linha formada pelos dois pontos para dividir o conjunto em dois subconjuntos de pontos, que serão processados de forma recursiva. \\
	Determine o ponto acima da linha com a distância máxima da linha. Este ponto forma um triângulo com os dois pontos na linha. \\
	Os pontos dentro desse triângulo não podem ser parte do casco convexo e, portanto, podem ser ignorados nos próximos passos. \\
	Repita recursivamente os dois passos anteriores nas duas linhas formadas pelos dois novos lados do triângulo. \\
	Repete-se o processo recursivamente para as novas regiões formadas, descartando pontos interiores \\
	O processo termina quando todos os subconjuntos estão vazios.
	\label{al:quickhull_2d}
\end{algorithm}

O Quickhull apresenta, em média, complexidade $O(n \log n)$ em duas dimensões, embora em casos degenerados possa chegar a $O(n^2)$. Em três dimensões, o algoritmo é adaptado para construir uma casca poliedral usando o mesmo princípio recursivo, consulte \citeonline{barber1996quickhull} para melhores definições.
