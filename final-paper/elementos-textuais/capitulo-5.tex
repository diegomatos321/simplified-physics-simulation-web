\chapter{Otimizações}

A eficiência computacional é um dos fatores determinantes para o desempenho de um motor de física em tempo real. Em jogos, animações interativas, simulações físicas e aplicações gráficas, a necessidade de atualizações contínuas e a obrigatoriedade de operar dentro de limites rigorosos de tempo tornam indispensável o uso de técnicas de otimização em todos os estágios do pipeline de simulação.

Como qualquer objeto pode potencialmente colidir com qualquer outro, uma simulação contendo $n$ objetos requer $(n-1)+(n-2)+\dots+1 = n(n-1)/2 = O(n^2)$ testes de pares no pior caso. Devido à complexidade quadrática, testar ingenuamente cada par torna-se impraticável mesmo para valores moderados de $n$.

Reduzir o custo associado ao teste de pares afetará o tempo de execução apenas linearmente. Para realmente acelerar o processo, o número de pares testados deve ser reduzido. Essa redução é realizada separando o tratamento de colisões de múltiplos objetos em duas fases: \textit{Narrow Phase} e \textit{Broad Phase}.

Neste capítulo, descrevemos as estratégias clássicas de otimização aplicadas à detecção e resolução de colisões, com foco na divisão entre \textit{Broad Phase} e \textit{Narrow Phase}, no uso de estruturas espaciais, na adoção de passos temporais fixos e na execução multi-thread de simulações físicas.

\section{O Fecho Convexo}

O Fecho Convexo (FC) de um conjunto é definido como a menor região convexa que contém todos os seus pontos, sendo frequentemente utilizado como um volume delimitador. Determinar o fecho convexo é um problema recorrente em computação geométrica, especialmente quando se deseja organizar pontos em estruturas mais simples ou acelerar operações posteriores, como testes de colisão.

Em sistemas de simulação física e detecção de colisões, a representação geométrica dos objetos influencia diretamente a eficiência dos cálculos. Formas complexas, com muitos vértices ou superfícies não convexas, tornam tais testes significativamente mais custosos. Uma solução comum é empregar aproximações convexas ou volumes delimitadores que possibilitam testes rápidos sem sacrificar excessivamente a precisão da simulação.

\begin{figure}[htb]
	\centering
	\includesvg[width=0.7\linewidth]{fecho-convexo-interseccao}
	\caption{As figuras à esquerda não tem intersecção de seus volumes delimitadores, logo não podem estar em colisão. Já as figuras da direita estão se sobrepondo, logo podem estar em colisão.}
	\label{fig:fecho-convexo-interseccao}
\end{figure}

A principal motivação para o uso de volumes delimitadores é que formas mais simples (como caixas ou esferas) permitem testes de sobreposição muito mais baratos do que a geometria original que envolvem, como ilustrado na figura~\ref{fig:fecho-convexo-interseccao}. Dessa forma, o FC atua como um primeiro filtro: apenas quando o teste de interseção entre volumes delimitadores retorna positivo é que se procede para verificações mais detalhadas na geometria original. Em muitos casos, o próprio volume delimitador já é suficiente para caracterizar uma colisão.

Segundo \citeonline{moller2018}, nem todos os objetos geométricos servem como volumes delimitadores eficazes. As propriedades desejáveis para volumes delimitadores incluem:
\begin{itemize}
	\item Testes de interseção de baixo custo
	\item Ajuste preciso
	\item Cálculo econômico
	\item Fácil de girar e transformar
	\item Consome pouca memória
\end{itemize}
como ilustrado na figura~\ref{fig:formas-fecho-convexo}.

\begin{figure}[htb]
	\caption{Comparação entre formas de fecho convexo}
	\centering
	\includegraphics[width=\textwidth]{bounding-volumes.png}
	\label{fig:formas-fecho-convexo}
\end{figure}

\subsection*{Caixa Delimitadora Alinhada ao Eixo Coordenado (AABB)}

A caixa delimitadora alinhada aos eixos (AABB) é um dos volumes delimitadores mais utilizados. Trata-se de um paralelepípedo (ou retângulo, em 2D) cujas faces são paralelas aos eixos do sistema de coordenadas. Seu teste de interseção é extremamente como no algoritmo~\ref{alg:interseccao-aabb}.
\begin{algorithm}[htb]
	\caption{Teste de Intersecção AABB}
	\KwIn{A e B: volumes AABB}
	\KwOut{Verdadeiro se houver colisão}
	
	\If{$A.x_{max} < B.x_{min}$ \textbf{ou} $A.x_{min} > B.x_{max}$}{
		\Return{False}
	}
	\If{$A.y_{max} < B.y_{min}$ \textbf{ou} $A.y_{min} > B.y_{max}$}{
		\Return{False}
	}
	\If{$A.z_{max} < B.z_{min}$ \textbf{ou} $A.z_{min} > B.z_{max}$}{
		\Return{False}
	}
	\Return{True}
	\label{alg:interseccao-aabb}
\end{algorithm}
AABBs são eficientes, porém perdem precisão quando o objeto sofre rotações, pois a caixa permanece alinhada aos eixos globais.

\subsection*{Caixa Orientada (OBB)}

Uma Caixa Orientada (OBB) é uma caixa retangular que pode estar arbitrariamente rotacionada em relação aos eixos do sistema de coordenadas. É definida por um ponto central $c$, por três vetores ortogonais $\hat{u}_i$ que compõem sua orientação e por semi-extensões $e_i$:

$$
	OBB = \left\{ c + \sum_{i=1}^{3} \alpha_i \hat{u}_i \; \middle| \; -e_i \le \alpha_i \le e_i \right\}.
$$

OBBs geralmente oferecem melhor ajuste, especialmente para objetos alongados ou rotacionados, reduzindo falsos positivos. Porém, o teste de interseção é mais caro que o das AABBs, o que torna seu uso preferível em cenas com número reduzido de objetos ou em simulações nas quais a precisão de ajuste é particularmente importante.

\subsection*{Esferas e Elipsoides}

As esferas constituem o volume delimitador mais simples, definidas apenas por um centro $c$ e um raio $r$:
$$
	\text{Sphere} = \{ x \in \mathbb{R}^3 \; | \; \|x - c\| \le r \}.
$$
Testes de colisão entre esferas são extremamente rápidos, porém inadequados para objetos de proporções irregulares. Elipsoides oferecem melhor ajuste, mas aumentam o custo de teste. Por isso, essas formas são frequentemente utilizadas em fases preliminares da detecção, ou como nós intermediários em hierarquias de volumes delimitadores (BVH).

\subsection*{Quickhull}

O \textit{Quickhull} é um algoritmo para o cálculo do fecho convexo de um conjunto finito de pontos em qualquer dimensão, adotando uma estratégia de divisão e conquista semelhante ao \textit{quicksort} \cite{barber1996quickhull}.

O Quickhull parte de um conjunto de pontos $S$ e constrói o polígono (ou poliedro) convexo que os contém.  O processo para 2 dimensões pode ser descrito em linhas gerais como no algoritmo~\ref{alg:quickhull_2d}.
\begin{algorithm}[htb]
	\caption{Quickhull 2D}
	\LinesNumbered
	\SetAlgoLined
	\KwIn{Polígono Convexo}
	\KwOut{Lista dos vértices do fecho convexo}
	
	Encontre os pontos de menor e maior coordenada em $x$; eles pertencem ao fecho convexo.\\
	Use a linha formada pelos dois pontos para dividir o conjunto em dois subconjuntos de pontos, que serão processados de forma recursiva. \\
	Para cada subconjunto, encontre o ponto mais distante da linha; ele forma um triângulo que exclui pontos interiores.\\
	Repita recursivamente os dois passos anteriores nas duas linhas formadas pelos dois novos lados do triângulo. \\
	O processo termina quando todos os subconjuntos estão vazios.
	\label{alg:quickhull_2d}
\end{algorithm}
Apresenta complexidade média $O(n \log n)$ em 2D, podendo chegar a $O(n^2)$ em casos degenerados. Em 3D, adapta-se a construções poliedrais mais complexas, mantendo o mesmo princípio recursivo.

\section{Broad Phase}

A fase de \textit{Broad Phase} tem como objetivo descartar rapidamente pares de objetos que seguramente não estão colidindo. Segundo \citeonline{ericson2004real}, "nada é mais rápido do que não ter que realizar uma tarefa". Portanto, a melhor otimização é reduzir a quantidade de trabalho o mais cedo possível.

Como objetos só podem colidir com outros que estejam fisicamente próximos, a \textit{Broad Phase} utiliza consultas espaciais para identificar apenas aqueles que compartilham regiões semelhantes do espaço. O teste mais simples usado nesta etapa é o teste de interseção booleana entre volumes delimitadores primitivos como no algoritmo~\ref{alg:broad-phase}, devido ao seu baixo custo computacional.

\begin{algorithm}[htb]
	\caption{Broad Phase generalizada}
	\KwOut{Pares de objetos potenciais para colisão}
	vistos $\leftarrow \{\}$ \\
	pares\_contato $\leftarrow \{\}$ \\
	\ForEach{bodyA em bodies}{
		candidates $\leftarrow$ consultar objetos próximos de bodyA \\
		\ForEach{bodyB em candidates}{
			\If{par \{bodyA, bodyB\} já foi visto}{
				continue
			}
			marcar par como visto \\
			\If{teste de interseção barata}{
				pares\_contato $\leftarrow$ \{bodyA, bodyB\}
			}
		}
	}
	\KwRet{pares\_contato}
	\label{alg:broad-phase}
\end{algorithm}

\subsection*{Grade uniforme}

As técnicas de partição do espaço é o processo pelo qual o espaço é subdividido em regiões convexas, chamadas células. Cada célula na partição mantém uma lista de referências a objetos que nela estão (parcialmente) contidos. Como os objetos só podem se interceptar se sobrepuserem à mesma região do espaço, o número de testes de pares de objetos é drasticamente reduzido.

Um esquema muito eficaz de subdivisão espacial consiste em sobrepor um espaço com uma grade regular. Essa grade divide o espaço em várias células de tamanho igual. Cada objeto é então associado às células com as quais se sobrepõem, como ilustrado na figura~\ref{fig:grade-uniforme}.

\begin{figure}[htb]
	\caption{Divisão espacial em grade uniforme}
	\centering
	\includegraphics[width=0.5\textwidth]{Partition-of-a-2D-scene-by-a-4x4-uniform-grid.png}
	\label{fig:grade-uniforme}
\end{figure}

Devido à uniformidade da grade, acessar uma célula correspondente a uma determinada coordenada é simples e rápido: os valores das coordenadas do mundo são simplesmente divididos pelo tamanho da célula para obter as coordenadas da célula. Dadas as coordenadas de uma célula específica, localizar as células vizinhas também é trivial.

Em termos de desempenho, um dos aspectos mais importantes dos métodos baseados em grade é a escolha de um tamanho de célula apropriado. Existem quatro questões relacionadas ao tamanho da célula que podem prejudicar o desempenho:
\begin{enumerate}
	\item Células muito pequenas geram atualizações excessivas.
	\item Células grandes demais fazem com que muitos objetos sejam agrupados, reduzindo a eficácia da Broad Phase.
	\item Objetos muito complexos demandam subdivisão para melhorar a qualidade dos testes.
	\item Cenários mistos exigem grades hierárquicas ou abordagens híbridas.
\end{enumerate}
O ideal é que cada objeto caiba exatamente no tamanho de uma célula.

\section{Narrow Phase}

Após a \textit{Broad Phase} reduzir a lista de pares, a \textit{Narrow Phase} realiza testes geométricos precisos. Esta fase utiliza algoritmos complexos como:
\begin{itemize}
	\item SAT (Separating Axis Theorem) para polígonos/poliedros convexos.
	\item GJK (Gilbert–Johnson–Keerthi) para formas convexas arbitrárias.
	\item EPA (Expanding Polytope Algorithm) para obtenção da profundidade de penetração.
\end{itemize}
A complexidade é reduzida aos pares realmente necessários, tipicamente em número linear no tamanho da cena.

\section{Física com passo de tempo fixo}

A forma mais ingênua de simular física é utilizar o tempo decorrido entre quadros como passo de simulação, como ilustrado no algoritmo~\ref{alg:fisica-passo-variavel}. Embora simples, isso introduz instabilidade numérica: simulações podem divergir em altas taxas de quadros, apresentar \textit{tunneling} e se comportar de maneira não determinística. Essa abordagem trás um passo variável que será executada mais rápido ou mais lento dependendo do computador do usuário.
\begin{algorithm}[htb]
	\caption{Simulação física com passo variável}
	tempo\_anterior $\leftarrow$ agora() \\
	\While{!sair}{
		tempo\_agora $\leftarrow$ agora() \\
		dt $\leftarrow$ tempo\_agora - tempo\_anterior \\
		tempo\_anterior $\leftarrow$ tempo\_agora \\
		Física(dt) \\
		Renderizar() 
	}
	\label{alg:fisica-passo-variavel}
\end{algorithm}

Para garantir estabilidade, utiliza-se um passo fixo de simulação e um acumulador de tempo como no algoritmo~\ref{alg:fisica-passo-fixo}.
\begin{algorithm}[htb]
	\caption{Simulação física com passo fixo}
	tempo passado $\leftarrow$ agora() \\
	fixed dt $\leftarrow$ $\frac{1}{50}$ \\
	acumulador $\leftarrow$ 0 \\
	\While{!sair}{
		tempo agora $\leftarrow$ agora() \\
		dt $\leftarrow$ tempo agora - tempo anterior \\
		tempo anterior $\leftarrow$ tempo agora \\
		acumulador $\leftarrow$ acumulador + dt
		
		\While{acumulador <= fixed dt}{
			Física(dt) \\
			acumulador $\leftarrow$ acumulador - fixed dt
		}
		
		Física(dt) \\
		Renderizar() 
	}
	\label{alg:fisica-passo-fixo}
\end{algorithm}

\section{Simulação física multi-thread}

Em programas interativos tradicionais a lógica, a física e a renderização são executadas em uma única \textit{thread}. Nesse modelo, a renderização só ocorre após a conclusão da etapa de física, e qualquer uma das etapas pode se tornar gargalo.

Para um sistema single-thread, a renderização só pode começar depois que a física tiver sido simulada, ou seja, é impossível renderizar antes que a simulação física seja feita, como ilustrado na figura~\ref{fig:diagrama-threaded}. Nos casos em que uma alta quantidade de cálculo é necessária para a simulação de física, a renderização seria atrasada e a simulação o gargalo, resultando em baixas taxas de quadros e falhas gráficas. O contrário também pode ocorrer: a renderização demorar resulta em um atraso na leitura da entrada do usuário e no processo de simulação física.

Para solucionar esse problema, a simulação física pode ser movida para um \textit{thread} dedicada. A \textit{thread} principal renderiza continuamente utilizando o estado físico mais recente, enquanto o \textit{thread} secundário calcula atualizações físicas em paralelo, como ilustrado na figura~\ref{fig:diagrama-threaded}.

\begin{figure}[htb]
	\centering
	\includesvg[width=\textwidth]{diagrama-threaded}
	\caption{À esquerda esquema tradicional single-thread. À direita esquema com simulação rodando numa thread dedicada a passo fixo.}
	\label{fig:diagrama-threaded}
\end{figure}

Essa separação permite:
\begin{itemize}
	\item melhor utilização de múltiplos núcleos;
	\item redução da latência na renderização;
	\item maior taxa de quadros mesmo em cenas fisicamente complexas;
	\item desacoplamento total entre física e renderização.
\end{itemize}
Dessa forma, a arquitetura é essencial em jogos modernos e simulações interativas, especialmente em ambientes Web utilizando \textit{Web Workers}.
