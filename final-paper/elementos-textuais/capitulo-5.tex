\chapter{Resposta a colisão}

A resposta a colisões é uma etapa fundamental em qualquer sistema de simulação física interativa. Após detectar que dois corpos estão em interpenetração, torna-se necessário aplicar um conjunto de correções que restaurem a plausibilidade física do movimento, evitando instabilidades numéricas. Neste capítulo discutimos os princípios clássicos, as formulações modernas e a relação direta entre métodos geométricos empregados na fase de detecção — como o vetor de translação mínima (MTV) — e métodos baseados em partículas e restrições, como o modelo proposto por \citeonline{jakobsen2001advanced}.

\section{Fundamentos da Resposta a Colisão}

A resposta a colisões consiste, essencialmente, em duas operações principais:

\begin{enumerate}
	\item \textbf{Correção de Posição}: eliminar a interpenetração entre dois corpos.
	\item \textbf{Correção de Velocidade}: remover ou ajustar componentes da velocidade que induziriam novo contato imediato.
\end{enumerate}

Embora os motores tradicionais realizem ambas as etapas muitos sistemas focam totalmente na correção de posições, derivando velocidades implicitamente a partir da diferença entre posições sucessivas.

\section{Métodos Dinâmicos na Simulação Física}

A literatura apresenta diversas abordagens para modelar o movimento de corpos rígidos e deformáveis. Embora este trabalho se baseie no método simplificado de Jakobsen, é importante contextualizar outras categorias amplamente utilizadas em motores físicos modernos:

\subsection*{Método do Impulso}

Os \textit{Métodos do Impulso} tratam colisões aplicando impulsos instantâneos que alteram diretamente as velocidades dos corpos para preservar o momento linear e angular. Essa abordagem é utilizada em motores como Havok e Bullet. O impulso é calculado em função da velocidade relativa no ponto de contato, resultando em um método eficiente para simulações em tempo real.

\subsection*{Método de Penalidades}

Nos \textit{Métodos de Penalidades}, colisões são tratadas como interpenetrações que geram forças de repulsão proporcionais à profundidade de penetração alterando diretamente a aceleração. Essas forças geralmente seguem modelos de mola e amortecimento, trata-se de um método simples, porém sensível à escolha dos parâmetros de rigidez, podendo causar instabilidade numérica.

\subsection*{Método Constraint-Based com Multiplicadores de Lagrange}

Os métodos baseados em restrições formulam os contatos como equações que devem ser satisfeitas exatamente. São resolvidos usando multiplicadores de Lagrange, essa abordagem é robusta e adequada para sistemas complexos, mas exige a solução de sistemas lineares, tornando sua aplicação onerosa em plataformas Web.

\section{Processo de Separação}

A metodologia proposta por \citeonline{jakobsen2001advanced}, embora descrita em termos de integração Verlet e relaxamento de restrições geométricas, pode ser fundamentalmente entendida como uma precursora direta das modernas abordagens de \textit{Position-Based Dynamics} (PBD), popularizadas por \citeonline{muller2007position}. 

Diferentemente de abordagens que atuam diretamente nas velocidades ou utilizam impulsos, Jakobsen trata colisões como restrições geométricas adicionais que devem ser satisfeitas durante o processo iterativo de relaxação da simulação. Assim, objetos penetrando um ao outro são corrigidos exclusivamente por modificações de posição, de forma estável e sem oscilações numéricas.

A resposta à colisão envolve dois passos principais. O primeiro consiste em separar os elementos geométricos (vértices, arestas ou faces) que se encontram em interpenetração, o que caracteriza um processo estritamente geométrico. O segundo passo corresponde a um processo iterativo de relaxamento, no qual os elementos afetados ajustam suas posições de acordo com as restrições impostas pelo sistema físico.

Para dois objetos convexos $A$ e $B$ em colisão, o esquema de detecção de colisão deve retornar os pontos de contato de cada objeto e o tamanho da penetração. Com essas informações devemos tratar duas configurações possíveis: ponto de contato pertence a um vértice ou pertence a uma aresta.

\textbf{INSERIR DIAGRAMA MOSTRANDO OS DOIS CASOS}

A separação não é aplicada a um ponto rígido, mas distribuída entre as partículas que definem os corpos. Denotando por $\vec{p}$ e $\vec{q}$ os pontos de contato estimados em $A$ e $B$, respectivamente, a correção por projeção é dividida igualmente entre os dois corpos:

\begin{equation}
	\Delta \vec{p} = -\frac{1}{2}\vec{d}, \qquad
	\Delta \vec{q} = +\frac{1}{2}\vec{d}.
\end{equation}

Caso o ponto de contato seja um vértice único, basta mover os vértice para fora da região inválida. Caso seja uma colisão tipo aresta-vértice o ponto de contato $p$ cai entre dois vértices $x_1$ e $x_2$ e o nosso objetivo é corrigir as suas posições para uma configuração válida $x_1^*$, $x_2^*$, logo pela equação da reta $p$ pode ser descrito como uma interpolação linear

\begin{equation}
	p = \alpha x_1 + (\alpha-1) x_2, \qquad 0 \leq \alpha \leq 1.
	\label{eq:parametric_edge}
\end{equation}

Durante a separação, a correção $\Delta \vec{p}$ deve alterar indiretamente $\vec{x}_1$ e $\vec{x}_2$ de forma proporcional a essa parametrização. A partir da Eq.~\ref{eq:parametric_edge}, derivamos o valor de $\alpha$ projetando o vetor $\vec{p} - \vec{x}_2$ sobre o vetor da aresta $\vec{x}_1 - \vec{x}_2$:

Dessa forma Jakobsen computa as novas posições movendo as partículas proporcionalmente a $\alpha$

\begin{align}
	x_1 += \alpha \Delta_p \\
	x_2 += (1-\alpha) \Delta_p \\
	\alpha
	= 
	\frac{
		(\vec{p} - \vec{x}_2) \cdot (\vec{x}_1 - \vec{x}_2)
	}{
		\|\vec{x}_1 - \vec{x}_2\|^2
	}.
	\label{eq:alpha_projection}
\end{align}

Isso garante que a geometria original é preservada e que o ponto $\vec{p}$, definido implicitamente pelos vértices da aresta, é deslocado exatamente pela quantidade desejada.

\section{Algoritmo de Expansão de Politopos (EPA)}

Para realizar a separação de dois objetos usando o algoritmo SAT basta calcularmos o MTV como visto na seção \ref{sec:sat}. Já para o GJK é preciso fazer um segundo passo, uma extensão do algoritmo que nos permite encontrar a normal correta e profundidade das colisões.

O Algoritmo de Expansão de Politopos (do inglês Expanding Polytope Algorithm, EPA) cria um polítopo (ou polígono) dentro da Diferença de Minkowski e iterativamente expandi-lo até atingirmos a borda da Diferença de Minkowski. EPA executa essa tarefa utilizando a mesma função de suporte utilizada nos demais algoritmos e a mesma noção de um simplex.

Este algoritmo é uma extensão porque sua entrada é o Simplex final do GJK que contém a origem e encontra o MTV. A distância entre o ponto mais próximo com a origem é a profundidade de penetração ($\delta$). Além disso, o vetor normal para o ponto mais próximo é a direção de separação (ponto de contato). A solução ingênua é usar o normal da face mais próxima da origem, porém um simplex não precisa conter nenhuma das faces do polígono original, o quê pode acabar com uma normal incorreta.

O algoritmo expande o Simplex adicionando vértices a ele até encontrarmos a normal mais próxima de uma face que está no polígono original.

\begin{algorithm}[H]
	\caption{EPA}
	\KwIn{Simplex}
	\KwOut{separation $v$, penetration $\delta$}
	
	\For{$i \leftarrow 0$ \KwTo $i < MAX\_ITERATION$}{
		e $\leftarrow$ Encontrar aresta mais próxima a origem \\
		p $\leftarrow$ Calcular novo ponto de suporte na direção da normal de e \\
		$\delta \leftarrow p \cdot normal(e)$ \\
		\If{$|\delta - length(e)| < TOLERANCE$}{
			\KwRet{normal(e), $\delta$}
		}
		
		Adicionar ponto ao simplex
	}
\end{algorithm}

É importante limitar o número de iterações para evitar que a rotina entre em loop infinito em casos degenerados, como esse algoritmo converge rapidamente uma constante igual a 30 é um bom limite superior. Matematicamente a distância deve ser igual a zero, mas por conta da artimética de ponto flutuante, uma tolerância pequena deve ser aceita, como $10^{-3}$.

\section{Limitações}
\label{sec:limitacoes}

Ao adotar métodos simplificados, abre-se mão de características essenciais de motores físicos completos. Entre as limitações mais relevantes estão:

\begin{itemize}
	\item \textbf{Ausência de conservação precisa de energia e momento}, o que reduz o realismo de certas interações.
	\item \textbf{Incapacidade de simular materiais complexos} (ex.: fricção anisotrópica, torques realistas, elasticidade avançada).
	\item \textbf{Dependência de parâmetros empíricos}, sem interpretação física clara.
	\item \textbf{Menor robustez para geometrias arbitrárias}, especialmente polígonos concavos ou mal escalonados.
	\item \textbf{Dificuldade de lidar com sistemas altamente conectados} (estruturas rígidas, máquinas, esqueletos).
\end{itemize}

Essas limitações não invalidam o uso das técnicas, mas reforçam a necessidade do uso da aplicação a cenários onde a prioridade é a responsividade, e não a precisão física.

\subsection*{Jittering}

Em sistemas baseados em posições a estabilidade depende fortemente do processo de correção de posições. Um dos problemas mais recorrentes é o \textbf{jitter}, um tremor ou oscilação indesejada no posicionamento dos corpos, especialmente perceptível quando múltiplas restrições são aplicadas simultaneamente ou quando o sistema é altamente rígido.

\subsection*{Empilhamento}

Métodos simplificados têm dificuldade em manter pilhas estáveis de objetos, principalmente quando as correções não são distribuídas de forma global e consistente. O empilhamento tende a "escorregar" ou colapsar devido à falta de amortecimento numérico adequado.

\subsection*{Tunneling}

Ocorre quando objetos em alta velocidade atravessam outros sem detectar colisão. Métodos baseados exclusivamente em detecção discreta apresentam maior risco, especialmente quando o passo temporal é grande ou a geometria é fina.

