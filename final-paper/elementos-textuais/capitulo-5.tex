\chapter{O Fecho Convexo}

O Fecho Convexo (FC) de um conjunto é definido como a menor região convexa que contenha todos os pontos (também chamado como volume delimitador). Achar o fecho convexo de um conjunto de pontos, por exemplo, é um problema que aparece quando queremos organizar o conjunto, agrupar os pontos em uma região simples. 

Em 2 dimensões, temos um polígono, que pode ser descrito por uma lista ordenada de seus vértices. Em 3 dimensões, temos um poliedro, que precisa de estruturas um pouco mais complicadas para armazenar as faces que compõem a fronteira. Em dimensões maiores, as coisas se complicam ainda mais, pois precisaríamos de estruturas que representassem as faces de todas as dimensões. 

\textbf{INSERIR FIGURA COM EXEMPLOS DE USO}

Testar diretamente a geometria de dois objetos para verificar se há colisão entre eles é frequentemente muito caro, especialmente quando os objetos consistem em centenas ou até milhares de polígonos. Para minimizar esse custo, o fecho convexo dos objetos geralmente são testados quanto à sobreposição antes que o teste de interseção geométrica seja realizado.

\textbf{INSERIR FIGURA DO FECHO CONVEXO DE OBJETOS EM INTERSECÇÃO E SEM INTERSECÇÃO}

A ideia é que formas mais simples (como caixas e esferas) tenham testes de sobreposição mais baratos do que os objetos complexos que eles delimitam. O uso do FC permite testes rápidos de rejeição de sobreposição, pois só é necessário testar a geometria complexa delimitada quando a consulta inicial de sobreposição para os volumes delimitadores resulta em um resultado positivo. Em algumas aplicações, o próprio teste de interseção dos volumes delimitadores serve como prova suficiente de colisão.

Segundo \citeonline{moller2018}, nem todos os objetos geométricos servem como volumes delimitadores eficazes. As propriedades desejáveis para volumes delimitadores incluem:

\begin{itemize}
	\item Testes de interseção de baixo custo
	\item Ajuste preciso
	\item Cálculo econômico
	\item Fácil de girar e transformar
	\item Consome pouca memória
\end{itemize}

\textbf{INSERIR IMAGENS COM TIPOS DIFERENTES FORMAS DE FECHO CONVEXO}

A ideia principal por trás dos volumes delimitadores é preceder testes geométricos complexos com testes menos dispendiosos que permitem que o teste seja interrompido precocemente. Para suportar testes de sobreposição de baixo custo, o volume delimitador deve ter uma forma geométrica simples. Ao mesmo tempo, para tornar o teste de corte antecipado o mais eficaz possível, o volume delimitador também deve ser o mais ajustado possível, resultando em um trade-off entre o ajuste e o custo do teste de interseção.

\section{Caixa Delimitadora Alinhada ao Eixo Coordenado (AABB)}

A caixa delimitadora mínima para um conjunto de pontos em N dimensões é a caixa com a menor medida (área, volume, ou hipervolume em dimensões superiores) possível que englobe todos os pontos. A caixa delimitadora mínima de um conjunto de pontos é a mesma que a caixa delimitadora mínima de seu FC, um fato que pode ser usado de forma heurística para acelerar a computação.

A caixa delimitadora alinhada aos eixos (em inglês "axis-aligned bounding box", ou AABB) é um dos volumes delimitadores mais comuns. É uma caixa retangular de seis lados (em 3D, quatro lados em 2D) sujeita à restrição de que as bordas da caixa devem ser paralelas aos eixos de coordenadas cartesianos.

\textbf{INSERIR IMAGEM DE AABB}

\begin{algorithm}
	\caption{Teste de Intersecção AABB}
	\KwIn{Objeto A, Objeto B}
	\KwOut{Boolean}
	\If{$x_{max}$ de A < $x_{min}$ de B ou $x_{min}$ de A > $x_{max}$ de B}{
		\KwResult{False}
	}
	\If{$x_{max}$ de A < $x_{min}$ de B ou $x_{min}$ de A > $x_{max}$ de B}{
		\KwResult{False}
	}
	\If{$x_{max}$ de A < $x_{min}$ de B ou $x_{min}$ de A > $x_{max}$ de B}{
		\KwResult{False}
	}
	\KwResult{True}
\end{algorithm}

\section{Quickhull}

Quickhull é um algoritmo incremental para Fecho Convexo de um conjunto finito de pontos de qualquer dimensão. Ele usa uma abordagem de divisão e conquista semelhante à do quicksort, da qual seu nome deriva \cite{barber1996quickhull}.

O algoritmo de 2 dimensões pode ser dividido nas seguintes etapas:

\begin{algorithm}[H]
	\caption{Quickhull 2D}
	\LinesNumbered
	\SetAlgoLined
	\KwIn{Polígono Convexo}
	\KwOut{Lista dos vértices que representam o fecho convexo}
	Encontre os pontos com coordenadas mínimas e máximas x, pois estes sempre farão parte do casco convexo. Se muitos pontos com o mesmo mínimo/máximo x existirem, use os com o mínimo/máximo y, respectivamente. \\
	Use a linha formada pelos dois pontos para dividir o conjunto em dois subconjuntos de pontos, que serão processados de forma recursiva. Em seguida, descrevemos como determinar a parte do casco acima da linha; a parte do casco abaixo da linha pode ser determinada de forma semelhante. \\
	Determine o ponto acima da linha com a distância máxima da linha. Este ponto forma um triângulo com os dois pontos na linha. \\
	Os pontos dentro desse triângulo não podem ser parte do casco convexo e, portanto, podem ser ignorados nos próximos passos. \\
	Repita recursivamente os dois passos anteriores nas duas linhas formadas pelos dois novos lados do triângulo. \\
	Continue até que não mais pontos sejam deixados, a recursão chegou ao fim e os pontos selecionados constituem o casco convexo.
	\label{al:quickhull_2d}
\end{algorithm}

Para dimensões maiores extrapola o escopo desse trabalho, consulte \citeonline{barber1996quickhull} para melhores definições.
