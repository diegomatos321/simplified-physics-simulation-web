\chapter{Resposta a colisão}

A resposta a colisões é uma etapa fundamental em qualquer sistema de simulação física interativa. Após detectar que dois corpos estão em interpenetração, torna-se necessário aplicar um conjunto de correções que restaurem a plausibilidade física do movimento, evitando instabilidades numéricas. 

Este capítulo discute os princípios clássicos e as formulações modernas de resposta a colisão, estabelecendo a relação entre os métodos geométricos de detecção e a abordagem baseada em partículas e restrições proposta por \citeonline{jakobsen2001advanced}, que fundamenta este trabalho.

\section{Métodos Dinâmicos na Simulação Física}

A resposta a colisões consiste, essencialmente, em duas operações principais:
\begin{enumerate}
	\item Correção de Posição: eliminar a interpenetração entre dois corpos.
	\item Correção de Velocidade: remover ou ajustar componentes da velocidade que induziriam novo contato imediato.
\end{enumerate}
A literatura apresenta diversas abordagens para modelar o movimento de corpos rígidos e deformáveis. Embora este trabalho se baseie no método simplificado de \citeonline{jakobsen2001advanced}, é importante contextualizar outras categorias amplamente utilizadas em motores físicos modernos:

\subsection*{Método do Impulso}

O Método do Impulso tratam colisões aplicando impulsos instantâneos que alteram diretamente as velocidades dos corpos para preservar o momento linear e angular. Essa abordagem é utilizada em motores como Havok~\cite{havok} e Bullet~\cite{bullet}. O impulso é calculado em função da velocidade relativa no ponto de contato, resultando em um método eficiente para simulações em tempo real.

\subsection*{Método de Penalidades}

Nos Métodos de Penalidades, colisões são tratadas como interpenetrações que geram forças de repulsão proporcionais à profundidade de penetração alterando diretamente a aceleração. Essas forças geralmente seguem modelos de mola e amortecimento, trata-se de um método simples, porém sensível à escolha dos parâmetros de rigidez, podendo causar instabilidade numérica.

\subsection*{Método Constraint-Based com Multiplicadores de Lagrange}

Os métodos baseados em restrições formulam os contatos como equações que devem ser satisfeitas exatamente. São resolvidos usando multiplicadores de Lagrange, essa abordagem é robusta e adequada para sistemas complexos, mas exige a solução de sistemas lineares, tornando sua aplicação onerosa em plataformas Web.

\section{Processo de Separação}

A metodologia de \citeonline{jakobsen2001advanced} pode ser compreendida como uma precursora da moderna \textit{Position Based Dynamics} (PBD)~\cite{muller2007position}. Diferentemente das abordagens que atuam sobre a velocidade ou aceleração, \citeonline{jakobsen2001advanced} trata colisões como restrições geométricas adicionais que devem ser satisfeitas durante o processo iterativo de relaxação da simulação. Assim, objetos penetrando um ao outro são corrigidos exclusivamente por modificações de posição, de forma estável e sem oscilações numéricas.

A resposta à colisão envolve dois passos principais. O primeiro consiste em separar os elementos geométricos (vértices, arestas ou faces) que se encontram em interpenetração, o que caracteriza um processo estritamente geométrico. O segundo passo corresponde a um processo iterativo de relaxamento, no qual os elementos afetados ajustam suas posições de acordo com as restrições impostas pelo sistema físico.

Para dois objetos convexos $A$ e $B$ em colisão, o esquema de detecção de colisão deve retornar os pontos de contato de cada objeto e o tamanho da penetração. Com essas informações devemos tratar duas configurações possíveis: colisão vértice-vértice ou colisão vértice-aresta.

\subsection*{Colisão Vértice-Vértice}

Se o contato ocorre pontualmente entre duas partículas $p$ e $q$, a correção é dividida igualmente (assumindo massas iguais):
\begin{equation}
	\Delta \vec{p} = -\frac{1}{2}\vec{d}, \qquad
	\Delta \vec{q} = +\frac{1}{2}\vec{d}.
\end{equation}

\subsection*{Colisão Vértice-Aresta}

\begin{figure}[htb]
	\centering
	\includesvg[width=0.6\linewidth]{processo-separacao}
	\caption{Processo de separação caso colisão vértice-vértice. Correção das posição para configuração válida. Perceba que como $p$ está mais próximo de $x_1$, esse vértice é movido mais}
	\label{fig:processo-separacao}
\end{figure}

Esta é a configuração mais comum. Um vértice $p$ de um corpo colide contra uma aresta definida pelas partículas $\vec{x}_1$ e $\vec{x}_2$ de outro corpo. O ponto de impacto $p$ cai entre dois vértices $\vec{x}_1$ e $\vec{x}_2$ e o nosso objetivo é corrigir as suas posições para uma configuração válida $\vec{x}_1'$, $\vec{x}_2'$ de tal forma que garanta uma coerência geométrica, como mostra na figura~\ref{fig:processo-separacao}. Logo, pela equação da reta, $p$ pode ser descrito como uma interpolação linear
\begin{equation}
	p = \alpha \vec{x}_1 + (1-\alpha) \vec{x}_2, \qquad 0 \leq \alpha \leq 1,
	\label{eq:parametric_edge}
\end{equation}
ou seja, quando $\alpha=0, p=x_2$, quando $\alpha=1, p=x_1$.

Durante a separação, a correção $\Delta \vec{p}$ deve alterar indiretamente $\vec{x}_1$ e $\vec{x}_2$ de forma proporcional a essa parametrização. A partir da Eq.~\ref{eq:parametric_edge}, derivamos o valor de $\alpha$ calculando quanto o ponto de contato $p$ está "ao longo" da aresta que vai do ponto $\vec{x}_2 \rightarrow \vec{x}_1$. Dessa forma, \citeonline{jakobsen2001advanced} computa as novas posições movendo as partículas proporcionalmente a $\alpha$:
\begin{align*}
	\vec{x}_1 +&= \alpha \Delta_p \\
	\vec{x}_2 +&= (1-\alpha) \Delta_p \\
	\alpha
	&= 
	\frac{
		(\vec{p} - \vec{x}_2) \cdot (\vec{x}_1 - \vec{x}_2)
	}{
		\|\vec{x}_1 - \vec{x}_2\|^2
	}.
	\label{eq:alpha_projection}
\end{align*}
Isso garante que a geometria original é preservada e que o ponto $\vec{p}$, definido implicitamente pelos vértices da aresta, é deslocado exatamente pela quantidade desejada.

\section{Algoritmo de Expansão de Politopos (EPA)}

Enquanto o SAT fornece naturalmente o MTV, o algoritmo GJK apenas informa se há interseção (retornando um \textit{simplex} interno à diferença de Minkowski). Para obter a profundidade e a normal da colisão necessárias para a resposta física, utiliza-se o Algoritmo de Expansão de Politopos (EPA).

O algoritmo~\ref{alg:epa} expande o \textit{simplex} final do GJK iterativamente até encontrar a fronteira da diferença de Minkowski mais próxima da origem. A distância entre o ponto mais próximo com a origem é a profundidade de penetração. Além disso, o vetor normal para o ponto mais próximo é a direção de separação (ponto de contato). A solução ingênua é usar a normal da face mais próxima da origem, porém um \textit{simplex} não precisa conter nenhuma das faces do polígono original, o quê pode acabar com uma normal incorreta.
\begin{algorithm}[htb]
	\caption{EPA}
	\KwIn{Simplex}
	\KwOut{$\hat{n}$, $\delta$}
	
	\While{iterações $<$ MAX\_ITER}{
		e $\leftarrow$ Encontrar aresta mais próxima a origem \\
		p $\leftarrow$ Calcular novo ponto de suporte na direção da normal de e \\
		$\delta \leftarrow p \cdot normal(e)$ \\
		\If{$|\delta - length(e)| < \epsilon$}{
			\KwRet{normal(e), $\delta$}
		}
		
		Adicionar ponto ao simplex
	}
	\label{alg:epa}
\end{algorithm}
A tolerância $\epsilon$ (ex: $10^{-4}$) e o limite de iterações são cruciais para evitar loops infinitos em casos de precisão numérica flutuante ou formas curvas aproximadas.

\section{Limitações}

Ao adotar métodos simplificados, abre-se mão de características essenciais de motores físicos completos. Entre as limitações mais relevantes estão:
\begin{itemize}
	\item Ausência de conservação precisa de energia e momento, o que reduz o realismo de certas interações.
	\item Incapacidade de simular materiais complexos (ex.: fricção anisotrópica, torques realistas, elasticidade avançada).
	\item Dependência de parâmetros empíricos, sem interpretação física clara.
	\item Menor robustez para geometrias arbitrárias, especialmente polígonos concavos ou mal escalonados.
	\item Dificuldade de lidar com sistemas altamente conectados (estruturas rígidas, máquinas, esqueletos).
\end{itemize}
Essas limitações não invalidam o uso das técnicas, mas reforçam a necessidade do uso da aplicação a cenários onde a prioridade é a responsividade, e não a precisão física.

\subsection*{Jittering}

Em sistemas baseados em posições a estabilidade depende fortemente do processo de correção de posições. Um dos problemas mais recorrentes é o \textit{jitter}, um tremor ou oscilação indesejada no posicionamento dos corpos, especialmente perceptível quando múltiplas restrições são aplicadas simultaneamente ou quando o sistema é altamente rígido.

\subsection*{Empilhamento}

Métodos simplificados têm dificuldade em manter pilhas estáveis de objetos, principalmente quando as correções não são distribuídas de forma global e consistente. O empilhamento tende a "escorregar" ou colapsar devido à falta de precisão numérica adequado.

\subsection*{Tunneling}

Ocorre quando objetos em alta velocidade atravessam outros sem detectar colisão. Métodos baseados exclusivamente em detecção discreta apresentam maior risco, especialmente quando o passo temporal é grande ou a geometria é fina.

