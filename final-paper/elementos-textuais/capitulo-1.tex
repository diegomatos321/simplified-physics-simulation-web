\chapter{Introdução}

A evolução da capacidade computacional e das placas gráficas nas últimas décadas permitiu o desenvolvimento de simulações visuais cada vez mais realistas. Dentro desse contexto, a \textbf{Computação Gráfica} surge como uma área fundamental da ciência da computação responsável por estudar técnicas e algoritmos para gerar, manipular e representar imagens digitais de maneira eficiente e visualmente convincente. Essa área abrange desde o simples desenho de primitivas geométricas até a renderização de ambientes tridimensionais complexos, sendo amplamente utilizada em aplicações de engenharia, design, entretenimento e educação.

Segundo \citeonline{azevedo2003}, a computação gráfica é matemática e arte. Esta ferramenta proporciona um maior poder de abstração, ajudando na criação de imagens complexas e em muitos casos não imaginadas. A computação gráfica pode ser encarada como uma ferramenta não convencional que permite ao artista transcender das técnicas tradicionais de desenho ou modelagem.

Em paralelo, o avanço dos \textbf{programas interativos}, especialmente jogos eletrônicos e simulações físicas em tempo real, intensificou a necessidade de técnicas capazes de combinar realismo visual com desempenho computacional. De acordo com \citeonline{moller2018} num programa interativo, uma imagem é exibida, o usuário reage, e essa reação influencia as próximas imagens a serem geradas.
Esse ciclo de interação deve ocorrer a uma taxa suficientemente alta para que o usuário não veja imagens individuais, mas sim se sinta imerso em um processo dinâmico. A taxa na qual as imagens são exibidas é medida em quadros por segundo (FPS) ou Hertz (Hz). Para aplicações interativas, requer-se ao menos uma taxa de 6 FPS sendo que taxas mais elevadas tornam a experiência mais mais imersiva.

Tais programas exigem que o computador responda dinamicamente às ações do usuário, atualizando continuamente o estado do mundo virtual de acordo com as leis físicas e as interações entre objetos. Para atingir essa responsividade, é essencial empregar algoritmos eficientes para os problemas mais importantes \textit{detecção de colisões}, \textit{resposta física} e \textit{atualização de estados}.

\section{Contextualização do Problema}

Em animações tradicionais baseadas em quadros-chave (\textit{keyframe animation}), o movimento dos objetos é previamente definido, o que limita a capacidade de gerar comportamentos emergentes e interações físicas naturais. Para alcançar resultados mais dinâmicos, recorre-se à \textbf{animação baseada em física}, onde forças, restrições e colisões determinam o movimento de forma simulada. Contudo, implementar esse tipo de sistema envolve desafios significativos: é necessário equilibrar precisão física e desempenho computacional, especialmente porque a detecção de colisões — responsável por identificar interpenetrações e manter a consistência física da simulação — tende a ser a etapa mais custosa. Em cada instante de tempo, o sistema deve atualizar velocidades, forças e demais grandezas físicas, o que, em cenários com muitos objetos, dificulta a execução em tempo real.

Para lidar com essa complexidade, diversas técnicas são empregadas na literatura, como a \textbf{detecção de colisões em múltiplas fases} (Broad Phase e Narrow Phase), o uso de \textbf{estruturas espaciais otimizadas} (como grades uniformes) e estratégias de \textbf{processamento paralelo}. Além disso, diferentes abordagens de resposta física são utilizadas, como o método do impulso, o método da penalização e métodos baseados em restrições.

\section{Motores Físicos e Evolução Histórica}

Motores físicos comerciais e de código aberto moldaram a evolução das técnicas de simulação em tempo real. O \textbf{Havok} popularizou o uso profissional de física em jogos AAA, oferecendo um sistema robusto de colisões e restrições. O \textbf{NVIDIA PhysX} introduziu aceleração por GPU, permitindo simulações mais ricas. Motores como \textbf{Bullet Physics} e \textbf{Box2D}, ambos de código aberto, democratizaram o acesso a ferramentas de alta qualidade, tornando-se amplamente adotados em pesquisas, jogos independentes e aplicações embarcadas.

Essas ferramentas influenciaram profundamente metodologias modernas, e muitos dos algoritmos estudados neste trabalho derivam ou são inspirados em conceitos consolidados por esses motores.

\section{Aplicações Web e Motivação Tecnológica}

O desenvolvimento moderno de aplicações interativas na Web se beneficia da combinação de \textbf{WebGL}, \textbf{JavaScript} e \textbf{Web Workers}. WebGL proporciona renderização acelerada por GPU diretamente no navegador, enquanto JavaScript garante ampla acessibilidade e rápida prototipação. A utilização de Web Workers permite distribuir a simulação física em paralelo, evitando bloqueios na \textit{main thread} e mantendo o desempenho da renderização.

Esse ecossistema torna a plataforma Web um ambiente cada vez mais relevante para jogos, simulações científicas e aplicações educacionais, motivando o foco deste trabalho em uma solução física simplificada e otimizada para navegadores.

\section{Justificativa}

O estudo e a implementação de um sistema de animação física simplificada representam uma oportunidade de unir teoria e prática em Computação Gráfica e Física Computacional. Além de contribuir para a compreensão de conceitos fundamentais, um motor físico otimizado possui aplicações diretas em jogos digitais, visualização científica, engenharia e realidade virtual.

\citeonline{jakobsen2001advanced} propôs um esquema simplificado de simulação física, capaz de modelar objetos rígidos e deformáveis sem calcular torques ou tensores de inércia explicitamente. Seu método combina um integrador de Verlet, manutenção iterativa de restrições, resposta a colisões por projeção e uso de aproximações eficientes.

Entretanto, o método omite diversos detalhes importantes, como estratégias de detecção de colisões, tratamento de um grande número de restrições e mecanismos de otimização para múltiplos objetos — lacunas que este trabalho busca explorar.

\section{Objetivos}

Neste trabalho, o objetivo é desenvolver um protótipo inspirado no método de Jakobsen, apresentando soluções para detecção e resposta a colisões com foco em aplicações Web. São metas específicas:

\begin{itemize}
	\item Revisar os principais conceitos de animação baseada em física e integração numérica;
	\item Implementar algoritmos de detecção de colisão
	\item Desenvolver uma simulação física baseada em partículas e restrições utilizando o método de Jakobsen;
	\item Aplicar técnicas de otimização e processamento multi-threaded;
	\item Avaliar o desempenho e a estabilidade do sistema em diferentes cenários.
\end{itemize}

\section{Estrutura do Trabalho}

O Capítulo 2 apresenta os conceitos fundamentais da animação baseada em física, servindo como alicerce para a simulação. O Capítulo 3 detalha a abordagem central do trabalho, o método simplificado de Jakobsen, focando no integrador de Verlet e no relaxamento iterativo de restrições. O Capítulo 4 descreve os algoritmos essenciais para a detecção de colisões, como o SAT e o GJK. Em seguida, o Capítulo 5 explica as técnicas de resposta a colisão, incluindo a projeção de posição e o algoritmo EPA. O Capítulo 6 foca nas estratégias de otimização para garantir o desempenho em tempo real, cobrindo as fases Broad Phase e Narrow Phase, o uso de volumes delimitadores e o processamento multi-thread. O Capítulo 7 apresenta a metodologia experimental, a avaliação do sistema e a discussão dos resultados obtidos. Por fim, o Capítulo 8 resume as contribuições do trabalho, discute limitações e propõe direções para trabalhos futuros
