\chapter{Introdução}

A evolução da capacidade computacional e das placas gráficas nas últimas décadas permitiu o desenvolvimento de simulações visuais cada vez mais realistas. Dentro desse contexto, a \textbf{Computação Gráfica} surge como uma área fundamental da ciência da computação responsável por estudar técnicas e algoritmos para gerar, manipular e representar imagens digitais de maneira eficiente e visualmente convincente. Essa área abrange desde o simples desenho de primitivas geométricas até a renderização de ambientes tridimensionais complexos, sendo amplamente utilizada em aplicações de engenharia, design, entretenimento e educação.

Segundo \citeonline{azevedo2003}, a computação gráfica é matemática e arte. Esta ferramenta proporciona um maior poder de abstração, ajudando na criação de imagens complexas e em muitos casos não imaginadas. A computação gráfica pode ser encarada como uma ferramenta não convencional que permite ao artista transcender das técnicas tradicionais de desenho ou modelagem.

Em paralelo, o avanço dos \textbf{programas interativos}, especialmente jogos eletrônicos e simulações físicas em tempo real, intensificou a necessidade de técnicas capazes de combinar realismo visual com desempenho computacional. De acordo com \citeonline{moller2018}, renderização em tempo real refere-se à criação rápida de imagens no computador. É a área mais interativa da computação gráfica: uma imagem é exibida, o usuário reage, e essa reação influencia as próximas imagens a serem geradas.

Esse ciclo de interação deve ocorrer a uma taxa suficientemente alta para manter a sensação de fluidez, medida em quadros por segundo (FPS). A partir de aproximadamente 6 FPS já é possível perceber interatividade, e taxas mais elevadas tornam a experiência imersiva.

Tais programas exigem que o computador responda dinamicamente às ações do usuário, atualizando continuamente o estado do mundo virtual de acordo com as leis físicas e as interações entre objetos. Para atingir essa responsividade, é essencial empregar algoritmos eficientes de \textit{detecção de colisões}, \textit{resposta física} e \textit{atualização de estados}.

\section{Contextualização do Problema}

Em animações tradicionais baseadas em quadros-chave (\textit{keyframe animation}), o movimento dos objetos é previamente definido, o que limita a capacidade de gerar comportamentos emergentes e interações físicas naturais. Para alcançar resultados mais dinâmicos, recorre-se à \textbf{animação baseada em física}, onde forças, restrições e colisões determinam o movimento de forma simulada. Contudo, implementar esse tipo de sistema envolve desafios significativos: é necessário equilibrar precisão física e desempenho computacional, especialmente porque a detecção de colisões — responsável por identificar interpenetrações e manter a consistência física da simulação — tende a ser a etapa mais custosa. Em cada instante de tempo, o sistema deve atualizar velocidades, forças e demais grandezas físicas, o que, em cenários com muitos objetos, dificulta a execução em tempo real.

Para lidar com essa complexidade, diversas técnicas são empregadas na literatura, como a \textbf{detecção de colisões em múltiplas fases} (Broad Phase e Narrow Phase), o uso de \textbf{estruturas espaciais otimizadas} (como grades uniformes) e estratégias de \textbf{processamento paralelo}. Além disso, diferentes abordagens de resposta física são utilizadas, como o método do impulso, o método da penalização e métodos baseados em restrições.

\section{Métodos Dinâmicos na Simulação Física}

A literatura apresenta diversas abordagens para modelar o movimento de corpos rígidos e deformáveis. Embora este trabalho se baseie no método simplificado de Jakobsen, é importante contextualizar outras categorias amplamente utilizadas em motores físicos modernos: métodos \textit{impulse-based}, métodos \textit{penalty-based} e métodos \textit{constraint-based} via multiplicadores de Lagrange.

\subsection*{Método Método do Impulso}

Os métodos \textit{Método do Impulso} tratam colisões aplicando impulsos instantâneos que alteram diretamente as velocidades dos corpos para preservar o momento linear e angular. Essa abordagem é amplamente descrita em \citeonline{baraff1997rigid} e utilizada em motores como Havok e Bullet. O impulso é calculado em função da velocidade relativa no ponto de contato, resultando em um método eficiente para simulações em tempo real.

\subsection*{Método Metodo de Penalidades}

Nos métodos \textit{Metodo de Penalidades}, colisões são tratadas como interpenetrações que geram forças de repulsão proporcionais à profundidade de penetração. Essas forças geralmente seguem modelos de mola e amortecimento, como apresentado por \citeonline{wyvill1986soft}. Trata-se de um método simples, porém sensível à escolha dos parâmetros de rigidez, podendo causar instabilidade numérica.

\subsection*{Método Constraint-Based com Multiplicadores de Lagrange}

Os métodos baseados em restrições formulam os contatos como equações que devem ser satisfeitas exatamente. São resolvidos usando multiplicadores de Lagrange, como descrito em \citeonline{baraff1998large} e \citeonline{muller2007position}. Essa abordagem é robusta e adequada para sistemas complexos, mas exige a solução de sistemas lineares, tornando sua aplicação onerosa em plataformas Web.

\section{Motores Físicos e Evolução Histórica}

Motores físicos comerciais e de código aberto moldaram a evolução das técnicas de simulação em tempo real. O \textbf{Havok} popularizou o uso profissional de física em jogos AAA, oferecendo um sistema robusto de colisões e restrições. O \textbf{NVIDIA PhysX} introduziu aceleração por GPU, permitindo simulações mais ricas. Motores como \textbf{Bullet Physics} e \textbf{Box2D}, ambos de código aberto, democratizaram o acesso a ferramentas de alta qualidade, tornando-se amplamente adotados em pesquisas, jogos independentes e aplicações embarcadas.

Essas ferramentas influenciaram profundamente metodologias modernas, e muitos dos algoritmos estudados neste trabalho derivam ou são inspirados em conceitos consolidados por esses motores.

\section{Aplicações Web e Motivação Tecnológica}

O desenvolvimento moderno de aplicações interativas na Web se beneficia da combinação de \textbf{WebGL}, \textbf{JavaScript} e \textbf{Web Workers}. WebGL proporciona renderização acelerada por GPU diretamente no navegador, enquanto JavaScript garante ampla acessibilidade e rápida prototipação. A utilização de Web Workers permite distribuir a simulação física em paralelo, evitando bloqueios na \textit{main thread} e mantendo o desempenho da renderização.

Esse ecossistema torna a plataforma Web um ambiente cada vez mais relevante para jogos, simulações científicas e aplicações educacionais, motivando o foco deste trabalho em uma solução física simplificada e otimizada para navegadores.
%
%\section{Comparação entre Motores Físicos Clássicos}
%
%\begin{table}
%	\centering
%	\caption{Comparação entre motores físicos amplamente utilizados.}
%	\begin{tabular}{|l|c|c|c|c|}
%		\hline
%		\textbf{Característica} & \textbf{Havok} & \textbf{PhysX} & \textbf{Bullet} & \textbf{Box2D} \\
%		\hline
%		Dimensionalidade & 3D & 3D & 2D/3D & 2D \\
%		\hline
%		Licenciamento & Comercial & Gratuito/Proprietário & Open Source & Open Source \\
%		\hline
%		Modelo Dinâmico & Impulse/Constraint & Impulse/Constraint & Impulse/Constraint & Método do Impulso \\
%		\hline
%		GPU Acceleration & Parcial & Forte (CUDA) & Opcional & Não \\
%		\hline
%		Complexidade & Alta & Alta & Média & Baixa \\
%		\hline
%		Foco Principal & Jogos AAA & Simulações GPU e jogos & Pesquisa/Indie & Jogos 2D \\
%		\hline
%		Integração Web & Muito limitada & Limitada (via WASM) & Boa (via WASM) & Excelente \\
%		\hline
%	\end{tabular}
%	\label{tab:motores_fisicos}
%\end{table}

\section{Justificativa}

O estudo e a implementação de um sistema de animação física simplificada representam uma oportunidade de unir teoria e prática em Computação Gráfica e Física Computacional. Além de contribuir para a compreensão de conceitos fundamentais, um motor físico otimizado possui aplicações diretas em jogos digitais, visualização científica, engenharia e realidade virtual.

\citeonline{jakobsen2001advanced} propôs um esquema simplificado de simulação física, capaz de modelar objetos rígidos e deformáveis sem calcular torques ou tensores de inércia explicitamente. Seu método combina um integrador de Verlet, manutenção iterativa de restrições, resposta a colisões por projeção e uso de aproximações eficientes.

Entretanto, o método omite diversos detalhes importantes, como estratégias de detecção de colisões, tratamento de um grande número de restrições e mecanismos de otimização para múltiplos objetos — lacunas que este trabalho busca explorar.

\section{Objetivos}

Neste trabalho, o objetivo é desenvolver um protótipo inspirado no método de Jakobsen, apresentando soluções para detecção e resposta a colisões com foco em aplicações Web. São metas específicas:

\begin{itemize}
	\item Revisar os principais conceitos de animação baseada em física e integração numérica;
	\item Implementar algoritmos de detecção de colisão
	\item Desenvolver uma simulação física baseada em partículas e restrições utilizando o método de Jakobsen;
	\item Aplicar técnicas de otimização e processamento multi-threaded;
	\item Avaliar o desempenho e a estabilidade do sistema em diferentes cenários.
\end{itemize}

\section{Estrutura do Trabalho}

O Capítulo 2 apresenta conceitos fundamentais de animação baseada em física. O Capítulo 3 detalha o método simplificado de Jakobsen. O Capítulo 4 discute metodologias de detecção de colisões. O Capítulo 5 aborda o cálculo das fronteiras simplificadas dos objetos. O Capítulo 6 descreve o modelo de resposta a colisões via projeção de posições. O Capítulo 7 apresenta otimizações empregadas no sistema. O Capítulo 8 discute experimentos e comparações com outras bibliotecas. Por fim, o Capítulo 9 traz considerações finais e trabalhos futuros.
