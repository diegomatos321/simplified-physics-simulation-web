% ---
% Arquivo com a formatação do TCC.
% Este arquivo não deve ser modificado.
% ---
\usepackage[utf8]{inputenc}
\usepackage[T1]{fontenc}
\usepackage{amsmath}
\usepackage{amssymb,amsfonts,textcomp}
\usepackage{color}
\usepackage{array}
\usepackage{supertabular}
\usepackage{listings}         % Para as linguagens de programação
\usepackage{lastpage}		  % Usado pela Ficha catalográfica
\usepackage{indentfirst}	  % Indenta o primeiro parágrafo de cada seção.
\usepackage{hhline}
\usepackage{hyperref}
\usepackage[pdftex]{graphicx}
\graphicspath{ {./figuras/} }

% retira as mensagens de aviso do pacote Glossaries
\let\printglossary\relax
\let\theglossary\relax
\let\endtheglossary\relax
% coloca Seções em maiúsculas sem negrito no Sumário
\makeatletter
\let\oldcontentsline\contentsline
\def\contentsline#1#2{%
  \expandafter\ifx\csname l@#1\endcsname\l@section
    \expandafter\@firstoftwo
  \else
    \expandafter\@secondoftwo
  \fi
  {%
    \oldcontentsline{#1}{\normalfont\MakeTextUppercase{#2}}%
  }{%
    \oldcontentsline{#1}{#2}%
  }%
}
\makeatother
% ---
% Pacotes glossaries
% ---
\usepackage[subentrycounter,seeautonumberlist,nonumberlist=true]{glossaries}
% para usar o xindy ao invés do makeindex:
%\usepackage[xindy={language=portuguese},subentrycounter,seeautonumberlist,nonumberlist=true]{glossaries}
% ---
% Citações de referências no formato alfabético e negrito
\usepackage[alf, abnt-emphasize=bf]{abntex2cite} 
% Margens definidas em 25 mm para uso como documento em PDF
% Para imprimir use as seguintes margens:
% \usepackage[left=30mm, top=30mm, right=20 mm, bottom=20mm] {geometry}

\usepackage[margin=25 mm]{geometry}
%---
% O arquivo com o nome dos alunos e dos orientadores é lido aqui.
% Atualizar diretamente no arquivo
%---
%%%%%%%%%%%%%%%%%%%%%%%%%%%%%%%%%%%%%%%%%%%%%%%%%%%%%%%%%%%%%
% Definições de Macros utilizadas no TCC - ATUALIZE AQUI
%%%%%%%%%%%%%%%%%%%%%%%%%%%%%%%%%%%%%%%%%%%%%%%%%%%%%%%%%%%%%
\instituicao{UNIVERSIDADE FEDERAL DO RIO DE JANEIRO
\par
INSTITUTO DE COMPUTAÇÃO
\par
CURSO DE BACHARELADO EM CIÊNCIA DA COMPUTAÇÃO}
\title{Simulação Física Simplificada em Web}
\autor{Diego Vasconcelos Schardosim de Matos}
\orientador{Prof. Cláudio Esperança}
% \coorientador{Profa. Maria da Penha}
\local{RIO DE JANEIRO}
\data{2025}
\preambulo{Trabalho de conclusão de curso de graduação apresentado ao Instituto de Computação da Universidade Federal do Rio de Janeiro como parte dos requisitos para obtenção do grau de Bacharel em Ciência da Computação.}

%%%%%%%%%%%%%%%%%%%%%%%%%%%%%%%%%%%%%%%%%%%%%%%%%%%%%%%%%%%
% Corrige a fonte dos capítulos, seções, resumos, etc.
%%%%%%%%%%%%%%%%%%%%%%%%%%%%%%%%%%%%%%%%%%%%%%%%%%%%%%%%%%%

\renewcommand{\ABNTEXchapterfont}{\bfseries \rmfamily}  % Capítulos em Bold e Maiúsculas
\renewcommand{\ABNTEXchapterfontsize}{\normalsize}
\renewcommand{\ABNTEXsectionfont}{\rmfamily}            %  Seções em  Maiúsculas apenas
\renewcommand{\ABNTEXsectionfontsize}{\normalsize}
\renewcommand{\ABNTEXsubsectionfont}{\bfseries}         % Subseções em Bold apenas
\renewcommand{\ABNTEXsubsectionfontsize}{\normalsize}
\renewcommand{\lstlistingname}{Código}                  % Nome para os códigos no texto
\renewcommand{\lstlistlistingname}{Lista de \lstlistingname s}
\makeatletter                                          % Configura a linha da lista de códigos
\renewcommand\l@lstlisting[2]{{\normalfont\@dottedtocline{1}{1.5em}{2em}{Código~#1}{#2}}}
\makeatother
\usepackage{url16023}  % para retirar < e > da URL nas referências.
%%%%%%%%%%%%%%%%%%%%%%%%%%%%%%%%%%%%%%%%%%%%%%%%%%%%%%%%%%%
% Criação de quadros com numeração
%%%%%%%%%%%%%%%%%%%%%%%%%%%%%%%%%%%%%%%%%%%%%%%%%%%%%%%%%%%
\newcommand{\quadroname}{Quadro}
\newcommand{\listofquadrosname}{Lista de quadros}

\newfloat[chapter]{quadro}{loq}{\quadroname}
\newlistof{listofquadros}{loq}{\listofquadrosname}
\newlistentry{quadro}{loq}{0}
%%%%%%%%%%%%%%%%%%%%%%%%%%%%%%%%%%%%%%%%%%%%%%%%%%%%%%%%%%%
% configurações para atender às regras da ABNT
%%%%%%%%%%%%%%%%%%%%%%%%%%%%%%%%%%%%%%%%%%%%%%%%%%%%%%%%%%%
\setfloatadjustment{quadro}{\centering}
\counterwithout{quadro}{chapter}
\renewcommand{\cftquadroname}{\quadroname\space} 
\renewcommand*{\cftquadroaftersnum}{\hfill--\hfill}
%---
% Configuração de posicionamento padrão
%---
\setfloatlocations{quadro}{hbtp}
%---
% Para ajudar nas tabelas e quadros
%---
\makeatletter
\newcommand\arraybslash{\let\\\@arraycr}
\makeatother
%%%%%%%%%%%%%%%%%%%%%%%%%%%%%%%%%%%%%%%%%%%%%%%%%%%%%%%
% Redefine a macro para imprimir a capa 
%%%%%%%%%%%%%%%%%%%%%%%%%%%%%%%%%%%%%%%%%%%%%%%%%%%%%%%
\renewcommand{\imprimircapa}{%
\begin{capa}%
\center
\imprimirinstituicao
\par
\vspace*{1cm}
\MakeUppercase{\imprimirautor}
\vfill
\begin{center}
\imprimirtitulo
\end{center}
\vfill
\imprimirlocal
\par
\imprimirdata
\vspace*{1cm}
\end{capa}
}
%%%%%%%%%%%%%%%%%%%%%%%%%%%%%%%%%%%%%%%%%%%%%%%%%%%%%%%
% Redefine a macro para imprimir a folho de rosto
%%%%%%%%%%%%%%%%%%%%%%%%%%%%%%%%%%%%%%%%%%%%%%%%%%%%%%%
\renewcommand{\imprimirfolhaderosto}{%
\begin{capa}%
\center
\par
\vspace*{1cm}
\MakeUppercase{\imprimirautor}
\vfill
\begin{center}
\imprimirtitulo
\end{center}
\vfill
\hspace*{\fill}\parbox[b]{.5\textwidth}{%
        \linespread{1}\selectfont
\imprimirpreambulo
}
\vfill
\flushright
Orientador: \imprimirorientador\\

Co-orientador: \imprimircoorientador\\
\vfill
\begin{center}
\large\imprimirlocal
\par
\large\imprimirdata
\vspace*{1cm}
\end{center}
\end{capa}
}
%%%%%%%%%%%%%%%%%%%%%%%%%%%%%%%%%%%%%%%%%%%%%%%%%%%%%%%
% Informações do PDF inseridas automaticamente
% Atualizar apenas as palavras-chave se necessário
% Não modificar as cores dos links e referências
%%%%%%%%%%%%%%%%%%%%%%%%%%%%%%%%%%%%%%%%%%%%%%%%%%%%%%%
\makeatletter
\hypersetup{
pdftitle={\@title},
pdfauthor={\@author},
pdfsubject={\imprimirpreambulo},
pdfkeywords={MONOGRAFIA}{CIÊNCIA DA COMPUTAÇÃO}{DCC-UFRJ},
pdfcreator={LaTeX with abnTeX2},
colorlinks=true,
linkcolor=black,
citecolor=black,
urlcolor=black
}
\makeatother
%%%%%%%%%%%%%%%%%%%%%%%%%%%%%%%%%%%%%%%%%%%%%%%%%%%%
% Definição das Linguagens de Programação
%%%%%%%%%%%%%%%%%%%%%%%%%%%%%%%%%%%%%%%%%%%%%%%%%%%%
\definecolor{dkgreen}{rgb}{0,0.6,0}
\definecolor{gray}{rgb}{0.5,0.5,0.5}
\definecolor{purple}{rgb}{0.8,0,0.3}
\definecolor{orange}{rgb}{1,0.4,0}
\definecolor{lightlightgray}{rgb}{.95,.95,.95}
\definecolor{lightgray}{rgb}{.9,.9,.9}
\definecolor{lightgray2}{rgb}{.85,.85,.85}
\definecolor{darkgray}{rgb}{.4,.4,.4}
%---
% Comando para inserir a listagem de código, tem 4 parâmetros
% Linguagem, Caption, Label, Nome do Arquivo
%---
\newcommand{\includecode}[4][C]{\mbox{\lstinputlisting[caption=#2, label=#3, escapechar=, style=custom#1]{#4}}}

%---
%Define características em comum e numeração sequencial 
%---
\lstset{numberbychapter=false,framexleftmargin=5mm,  frame=shadowbox, rulesepcolor=\color{gray}}
%---
% A leitura da formatação das linguagens é feita aqui. 
% Modifique diretamente no arquivo.
%
%%%%%%%%%%%%%%%%%%%%%%%%%%%%%%%%%%%%%%%%%%%%%%%%%%%%
% Aqui você pode personalizar a linguagem C  
%%%%%%%%%%%%%%%%%%%%%%%%%%%%%%%%%%%%%%%%%%%%%%%%%%%%
\lstdefinestyle{customC}{
      language = C,
      breaklines=true,
      basicstyle=\footnotesize\ttfamily,
      keywordstyle=\bfseries\color{blue},
      commentstyle=\itshape\color{purple},
      identifierstyle=\color{black},
      stringstyle=\color{orange},
      showstringspaces=false}

%%%%%%%%%%%%%%%%%%%%%%%%%%%%%%%%%%%%%%%%%%%%%%%%%%%%
% Aqui você pode personalizar a linguagem Java 
%%%%%%%%%%%%%%%%%%%%%%%%%%%%%%%%%%%%%%%%%%%%%%%%%%%%       
\lstdefinestyle{customJava}{
      language = Java,
      breaklines=true,
      basicstyle=\footnotesize\ttfamily,
      keywordstyle=\bfseries\color{blue},
      commentstyle=\itshape\color{purple},
      identifierstyle=\color{black},
      stringstyle=\color{orange},
      showstringspaces=false}
%%%%%%%%%%%%%%%%%%%%%%%%%%%%%%%%%%%%%%%%%%%%%%%%%%%%
% Aqui você pode personalizar a linguagem C++
%%%%%%%%%%%%%%%%%%%%%%%%%%%%%%%%%%%%%%%%%%%%%%%%%%%%  
\lstdefinestyle{customC++}{
      language = C++,
      breaklines=true,
      basicstyle=\footnotesize\ttfamily,
      keywordstyle=\bfseries\color{blue},
      commentstyle=\itshape\color{purple},
      identifierstyle=\color{black},
      stringstyle=\color{orange},
      showstringspaces=false}
%%%%%%%%%%%%%%%%%%%%%%%%%%%%%%%%%%%%%%%%%%%%%%%%%%%%
% Aqui você pode personalizar a linguagem Javascript
%%%%%%%%%%%%%%%%%%%%%%%%%%%%%%%%%%%%%%%%%%%%%%%%%%%%        
\lstdefinestyle{customJavaScript}{
      language = JavaScript,
      breaklines=true,
      basicstyle=\footnotesize\ttfamily,
      keywordstyle=\bfseries\color{blue},
      commentstyle=\itshape\color{purple},
      identifierstyle=\color{black},
      stringstyle=\color{orange},
      showstringspaces=false}
      
% Javascript
\lstdefinelanguage{JavaScript}{
    aboveskip=15pt,
    keywords={typeof, new, true, false, catch, function, return, null, catch, switch, var, if, in, while, do, else, case, break},
    keywordstyle=\color{blue}\bfseries,
    ndkeywords={class, export, boolean, throw, implements, import, this},
    ndkeywordstyle=\color{darkgray}\bfseries,
    identifierstyle=\color{black},
    sensitive=false,
    comment=[l]{//},
    morecomment=[s]{/*}{*/},
    commentstyle=\color{purple}\ttfamily,
    stringstyle=\color{red}\ttfamily,
    morestring=[b]',
    morestring=[b]",
    rulecolor=\color{lightgray2},
    breaklines=true,
    basicstyle=\footnotesize\ttfamily,
    frame=single,
    backgroundcolor=\color{lightlightgray}
}
%%%%%%%%%%%%%%%%%%%%%%%%%%%%%%%%%%%%%%%%%%%%%%%%%%%%
% Aqui você pode personalizar o formato JSON
%%%%%%%%%%%%%%%%%%%%%%%%%%%%%%%%%%%%%%%%%%%%%%%%%%%%        
\lstdefinestyle{customJSON}{
      language = JSON,
      breaklines=true,
      basicstyle=\footnotesize\ttfamily,
      keywordstyle=\bfseries\color{blue},
      commentstyle=\itshape\color{purple},
      identifierstyle=\color{black},
      stringstyle=\color{orange},
      showstringspaces=false}
      
% JSON
\lstdefinelanguage{JSON}{
    aboveskip=15pt,
    string=[s]{"}{"},
    stringstyle=\color{blue},
    comment=[l]{:},
    commentstyle=\color{black},
    rulecolor=\color{lightgray2},
    breaklines=true,
    basicstyle=\footnotesize\ttfamily,
    frame=single,
    backgroundcolor=\color{lightlightgray}
}
%%%%%%%%%%%%%%%%%%%%%%%%%%%%%%%%%%%%%%%%%%%%%%%%%%%%
% Aqui você pode personalizar a linguagem de montagem do SAPIENS
%%%%%%%%%%%%%%%%%%%%%%%%%%%%%%%%%%%%%%%%%%%%%%%%%%%%       
\lstdefinestyle{customSapiens}{
      language = Sapiens,
      breaklines=true,
      basicstyle=\footnotesize\ttfamily,
      keywordstyle=\bfseries\color{blue},
      commentstyle=\itshape\color{purple},
      identifierstyle=\color{black},
      stringstyle=\color{orange},
      showstringspaces=false}

% Linguagem de Montagem    
\lstdefinelanguage{Sapiens}{
    aboveskip=15pt,
    keywordstyle=\color{blue}\bfseries,
    comment=[l]{;},
    commentstyle=\color{purple},
    rulecolor=\color{lightgray2},
    breaklines=true,
    basicstyle=\footnotesize\ttfamily,
    frame=single,
    backgroundcolor=\color{lightlightgray},
    language=[x86masm]Assembler,
    morekeywords={DB, DS, DW, JN, JSR, LDA, ORG, STA, STS, TRAP}
}
     
%---
% FIM
%---