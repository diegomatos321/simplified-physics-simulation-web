\begin{resumo}
	% Será que eu devo citar sobre objetos convexos e concâvos?
	\begin{SingleSpace}
		Neste trabalho é descrito elementos básicos para um esquema de modelagem baseado em física de objetos rígidos e deformáveis bem adequado para aplicações interativas que é simples, rápida e bastante estável. Estes corpos são representados por um grupo de partículas que devem satisfazer um conjunto de restrições lineares por um método de relaxamento e a simulação de seu movimento é usado um esquema de integração Verlet. A detecção das colisões são tratadas em duas fases: de \textit{Broad Phase} responsável por reduzir o número de candidatos à colisão com estruturas de divisão espacial e uso de testes rápidos e baratos; de \textit{Narrow Phase} responsável por usar algoritmos mais sofisticados para detectar colisão como o Teorema do Eixo Separador (Separating Axis Theorem - SAT) e algoritmo de \citeonline{gjk} (GJK). Para a resposta a colisão é usado as técnicas descritas por \citeonline{jakobsen2001advanced} pela projeção das posições das partículas envolvidas através do Vetor de Translação Mínimo (minimum translation vector - MTV). Diferente das abordagens tradicionais, a simulação física é obtida sem se computar explicitamente matrizes de orientação, torques ou tensores de inércia.
	\end{SingleSpace}
	\vspace{\onelineskip}
	\textbf{Palavras-chave}: Computação gráfica; Simulação Física; Detecção de Colisão; Resposta a Colisão; Corpos rígidos e deformáveis; web.
%latex. abntex. editoração de texto.
\end{resumo}
% Palavras-chave separadas e finalizadas por ponto


