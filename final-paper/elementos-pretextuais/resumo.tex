\begin{resumo}
	% Será que eu devo citar sobre objetos convexos e concâvos?
	\begin{SingleSpace}
		Este trabalho apresenta o desenvolvimento de um sistema de animação física simplificada para aplicações Web, fundamentado no método de \citeonline{jakobsen2001advanced}. O objetivo é investigar e demonstrar como técnicas de simulação leve podem produzir movimentos coerentes, estáveis e visualmente naturais, mesmo em ambientes com recursos computacionais limitados, como navegadores modernos. Para isso, são integradas abordagens clássicas de detecção e resposta a colisões, incluindo estratégias de Broad Phase e Narrow Phase, bem como métodos geométricos amplamente utilizados em sistemas interativos. Além disso, o projeto incorpora otimizações estruturais, como subdivisão espacial e processamento multi-threaded, buscando garantir escalabilidade e desempenho em cenários com múltiplos objetos dinâmicos. Os resultados obtidos evidenciam que é possível alcançar simulações eficientes e responsivas mantendo baixo custo computacional, tornando essas técnicas adequadas para jogos, visualizações interativas e aplicações educacionais na Web.
	\end{SingleSpace}
	\vspace{\onelineskip}
	\textbf{Palavras-chave}: Computação gráfica; Animação Física; Detecção de Colisão; Resposta a Colisão; Corpos rígidos e deformáveis; web.
%latex. abntex. editoração de texto.
\end{resumo}
% Palavras-chave separadas e finalizadas por ponto


