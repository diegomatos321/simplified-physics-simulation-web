\begin{resumo}[Abstract]
\begin{otherlanguage*}{english}
\begin{SingleSpace}
This work presents the development of a simplified physical animation system for Web applications, based on \citeonline{jakobsen2001advanced}’s method. The goal is to investigate and demonstrate how lightweight simulation techniques can produce coherent, stable, and visually natural motion, even in environments with limited computational resources, such as modern browsers. To achieve this, classical approaches to collision detection and response are integrated, including Broad Phase and Narrow Phase strategies, as well as geometric methods widely used in interactive systems. In addition, the project incorporates structural optimizations such as spatial subdivision and multi-threaded processing, aiming to ensure scalability and performance in scenarios with multiple dynamic objects. The results obtained show that it is possible to achieve efficient and responsive simulations while maintaining low computational cost, making these techniques suitable for games, interactive visualizations, and educational applications on the Web.
\end{SingleSpace}

%Eventually you can also write it in spanish \textit{(resumen}), french \textit{(résumé)}, italian \textit{(riassunto)} etc.

\vspace{\onelineskip}
   \textbf{Keywords}: Computer Graphics; Physics Animation; Collision Detection; Collision Response; Rigid and Deformable Bodies; Web.
   
 %  latex. abntex. text editoration.
 \end{otherlanguage*}
\end{resumo}




  
